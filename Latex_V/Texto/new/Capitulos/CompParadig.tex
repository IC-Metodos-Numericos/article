\chapter{Comparação entre Paradigmas de Programação}
Os métodos numéricos podem ser implementados utizando diferentes paradigmas de programação, como o imperativo e o funcional. No paradigma imperativo, o foco está na sequência de comandos que modificam o estado do programa, enquanto no paradigma funcional, a ênfase está na aplicação de funções e na imutabilidade dos dados. Ao implementar métodos numéricos, o paradigma imperativo pode ser mais intuitivo para aqueles familiarizados com a manipulação direta de variáveis e estruturas de controle, como loops e condicionais. Por outro lado, o paradigma funcional pode oferecer vantagens em termos de clareza e concisão, especialmente ao lidar com operações matemáticas complexas e recursivas. No entanto, a escolha do paradigma pode influenciar o desempenho e a legibilidade do código, dependendo do contexto e dos requisitos específicos do problema numérico em questão. Portanto, é importante considerar as características de cada paradigma ao implementar métodos numéricos para garantir eficiência e manutenção do código.

