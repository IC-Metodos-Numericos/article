\chapter{Problemas de Valor Inicial para Equações Diferenciais Ordinárias}

\section{Equações Diferenciais Ordinárias}
Muitos dos princípios por traz das leis da natureza são relações que estão diretamente ligados com a taxa de variação de certas quantidades. Matematicamente, essas relações são equações e as taxas de variação são representadas por derivadas. Equações que envolvem derivadas são chamadas de equações diferenciais, que em alguns casos são chamadas de Modelos Matemáticos. Equações diferenciais podem ser classificadas em dois tipos principais: equações diferenciais ordinárias (EDOs) e equações diferenciais parciais (EDPs).
Uma equação diferencial ordinária (EDO) é uma equação que envolve uma função desconhecida e suas derivadas em relação a uma única variável independente. As EDOs são amplamente utilizadas para modelar fenômenos em diversas áreas do conhecimento, como física, engenharia, biologia e economia. As EDOs podem ser classificadas de várias maneiras.

\section{Definição de Problema de Valor Inicial}

Um problema de valor inicial (PVI) para uma equação diferencial ordinária (EDO) consiste em encontrar uma função desconhecida $y(t)$ que satisfaça uma equação diferencial e que atenda a uma condição inicial especificada em um ponto $t_0$. Em termos gerais, um PVI pode ser formulado como:

\begin{equation}
    \begin{cases}
        y'(t) = f(t, y(t)), & t \in [a, b] \\
        y(t_0) = y_0
    \end{cases}
\end{equation}

onde $f$ é uma função dada, $y'(t)$ é a derivada de $y(t)$ em relação a $t$, e $y_0$ é o valor inicial da função no ponto $t_0$.

A solução de um PVI é uma função $y(t)$ que satisfaz tanto a equação diferencial quanto a condição inicial. A existência e unicidade de soluções para PVIs são garantidas sob certas condições, como as condições de Lipschitz e continuidade.

