\chapter{Problemas de Valor Inicial para Equações Diferenciais Ordinárias}

\section{Equações Diferenciais Ordinárias}
Muitos dos princípios por traz das leis da natureza são relações que estão diretamente ligados com a taxa de variação de certas quantidades. Matematicamente, essas relações são equações e as taxas de variação são representadas por derivadas. Equações que envolvem derivadas são chamadas de equações diferenciais, que em alguns casos são chamadas de Modelos Matemáticos. Equações diferenciais podem ser classificadas em dois tipos principais: equações diferenciais ordinárias (EDOs) e equações diferenciais parciais (EDPs).
Uma equação diferencial ordinária (EDO) é uma equação que envolve uma função desconhecida e suas derivadas em relação a uma única variável independente. As EDOs são amplamente utilizadas para modelar fenômenos em diversas áreas do conhecimento, como física, engenharia, biologia e economia. As EDOs podem ser classificadas de várias maneiras. Uma equação diferencial ordinária de primeira ordem é uma equação do tipo:
\begin{equation}
    y' = F\left(t, y\right)
\end{equation}
onde $t$ é a variável independente, $y$ é a função desconhecida, e
$y'$ é a derivada de $y$ em relação a $t$.

Exemplo:
\begin{equation}
    \frac{dy}{dt} = sen(t)
\end{equation}
A solução geral dessa EDO é dada por:
\begin{equation}
    y(t) = -cos(t) + C, \quad C \in \mathbb{R}
\end{equation}

\section{Definição de Problema de Valor Inicial}
Um problema de valor inicial (PVI) para uma equação diferencial ordinária consiste em encontrar uma função desconhecida $y(t)$ que satisfaça uma equação diferencial e que atenda a uma condição inicial especificada em um ponto $t_0$. Para a análise de existência e unicidade de soluções consideramos as condições de Lipschitz e continuidade. Em termos gerais, um PVI pode ser formulado como:

\begin{equation}
    \begin{cases}
        y'(t) = f(t, y(t)), & t \in [a, b] \\
        y(t_0) = y_0
    \end{cases}
\end{equation}

onde $f$ é uma função dada, $y'(t)$ é a derivada de $y(t)$ em relação a $t$, e $y_0$ é o valor inicial da função no ponto $t_0$.

A solução de um PVI é uma função $y(t)$ que satisfaz tanto a equação diferencial quanto a condição inicial. A existência e unicidade de soluções para PVIs são garantidas sob certas condições, como as condições de Lipschitz e continuidade. No geral, uma solução de uma EDO é uma família de funções, mas ao especificar uma condição inicial, obtemos uma solução única que passa pelo ponto inicial dado.
Exemplo:
\begin{equation}
    \frac{dy}{dt} = sen(t)
\end{equation}
Com a condição inicial:
\begin{equation}
    y(0) = 0
\end{equation}
A solução particular desse PVI é dada por:
\begin{equation}
    y(t) = -cos(t) + 1
\end{equation}

\section{Métodos Numéricos para Resolver PVIs}

Existem vários métodos numéricos para resolver problemas de valor inicial para equações diferenciais ordinárias. Neste capítulo, discutiremos alguns dos métodos mais comuns, incluindo o método de Euler, o método de Runge-Kutta.

\subsection{Teorema de Existência e Unicidade}
O teorema de existência e unicidade para PVIs estabelece condições sob as quais uma solução única existe para um problema de valor inicial. Essas condições geralmente envolvem a continuidade da função $f(t, y)$ e a satisfação da condição de Lipschitz em relação à variável $y$.

\begin{teo}[PVI]
    Suponha que a função $f(t, y)$ seja contínua em um retângulo $R = \{(t, y) | a \leq t \leq b, c \leq y \leq d\}$ e satisfaça a condição de Lipschitz em $y$ na mesma região. Então, para qualquer ponto inicial $(t_0, y_0)$ dentro de $R$, existe um intervalo $\left[t_0 - h, t_0 + h\right]$ em que o PVI 
    \begin{equation}
        \begin{cases}
            y'(t) = f(t, y(t)), & t \in [a, b] \\
            y(t_0) = y_0
        \end{cases}
    \end{equation}
    tem uma única solução em D.
\end{teo}

\subsection{Boa Colocação do PVI}
O problema 
\begin{equation}
    \begin{cases}
        y'(t) = f(t, y(t)), & t \in [a, b] \\
        y(t_0) = y_0
    \end{cases}
\end{equation}
é dito estar bem colocado se existe uma solução única que depende continuamente dos dados do problema, ou seja, da função $f$ e da condição inicial $y_0$. 
E para todo $\epsilon > 0$, existe um $\delta > 0$ tal que, se $||\tilde{f} - f|| < \delta$ e $|\tilde{y_0} - y_0| < \delta$, então a solução $\tilde{y}(t)$ do PVI perturbado satisfaz $||\tilde{y}(t) - y(t)|| < \epsilon$ para todo $t \in [a, b]$, uma solução única z(t) para o problema 
\begin{equation}
    \begin{cases}
        z'(t) = \tilde{f}(t, z(t)) + \delta(t) , & t \in [a, b] \\
        z(t_0) = \tilde{y_0}
    \end{cases}
\end{equation}

A boa colocação é uma propriedade desejável para garantir que pequenas variações nos dados do problema não resultem em grandes mudanças na solução.


\subsection{Método de Euler}
O método de Euler é um dos métodos numéricos mais simples para resolver PVIs. Ele é baseado na ideia de aproximar a solução da EDO usando uma série de passos pequenos.

