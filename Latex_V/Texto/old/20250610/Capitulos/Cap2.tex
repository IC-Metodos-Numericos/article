\chapter{Métodos Iterativos para Zeros de Função}

\section{Localização de Raízes}

\textcolor{blue}{Aqui, depois, fazer uma breve introdução sobre métodos para cálculo de raízes de funções, falando dos passos iniciais de determinação de intervalos que contenham raízes e em seguida do processo de refinamento.}  

\subsection{f(a)f(b) < 0}

\subsection{g(x) = h(x)}

Dada uma função, como $f(x) = x^3 -9x + 3$, é possível encontrar suas raízes $f(\xi) = 0$ analisando onde duas funções - g(x) e h(x) - construídas a partir delas se interceptam, ou seja, g(x) = h(x), $x^3 = 9x - 3$. \textcolor{blue}{Contar uma história do tipo "nem sempre é possível determinar raízes de funções de forma anlítica e por isso recorremos a métodos numéricos e bla bla bla".} %Evitar usar e.g., i.e.

\section{Método do Ponto Fixo}

\textcolor{blue}{Comece aqui a descrever o método do ponto fixo. Diga em linguagem escrita qual é a essência do MPF.} \textcolor{gray}{Achei interessante que tu que criou o título dessa seção, indicando que vc farejou a linha mental entre o parágrafo anterior e essa parte após esse comentário aqui em cinza}

Pode-se sistematizar essa prática construindo-se uma função de iteração na qual se isola x em um lado da igualdade. Usando o mesmo exemplo podemos obter, por exemplo:

\begin{multicols}{2}
\begin{itemize} %itemize (era enumerate)
    \item[a)] $\varphi_1(x) = \frac{x^3}{9} + \frac{1}{3}$
    \item[b)] $\varphi_2(x) = \sqrt[3]{9x-3}$
    \item[c)] $\varphi_3(x) = \frac{9}{x} - \frac{3}{x^2}$
    \item[d)] $\varphi_4(x) = \pm\sqrt{9 - \frac{3}{x}}$
    \item[e)] $\varphi_5(x) = x^3 - 8x + 3$
\end{itemize}
\end{multicols}
    
Todas essas funções tem um ponto fixo $\varphi(\xi) = \xi$ na raíz de f, $f(\xi) = 0$, o que significa que as interseções entre $\varphi(x)$ e a função identidade, são raízes $\xi$ da função original f(x). \textcolor{blue}{Enunciar de forma mais precisa essa equivalência, não está legal aqui.} \textcolor{orange}{Agora está legal?} \textcolor{blue}{Faça isso ser uma proposição. Faça isso depois de definir a fórmula geral de uma função de iteração $\varphi(x) = x + A(x)f(x)$} \textcolor{orange}{a proposição vem depois, junto com a fórmula geral, isto por 2 motivos que são, essencialmente, o mesmo: 1. Redundância; 2. Esse fato é uma exposição intuitiva que decorre do que já foi apresentado antes, a formalização sistemática com a prova vem depois como redundância.}

Todas essas funções podem ser adequadas a forma geral da função de iteração que é 
\begin{equation}
    \varphi(x) = x + A(x)f(x) \label{it}
\end{equation}
em que $A(\xi) \ne 0$. Podemos adequar, por exemplo, $\varphi_5$ da função $f(x) = x^3 -9x + 3$ a essa forma através da seguinte manipulação algébrica, começando por somar 0 (somando e subtraindo x):
\begin{align*}
    \varphi_5(x) &= x^3 - 8x + 3 \\
    x - x + \varphi_5(x) &= x - x + x^3 - 8x + 3 \\
    0 + \varphi_5(x) &= x + x^3 - 9x + 3\\
    \varphi_5(x) &= x + f(x)\\
    \varphi_5(x) &= x + (1)f(x)
\end{align*}
em que A(x) = 1. Sendo A(x) uma função constante, podemos perceber que ela nunca será nula.

%\section{Prova da equivalência}
\textcolor{blue}{Escreva algo do tipo ''vamos demonstrar o resultado da Proposição XXX". Use um ambiente do tipo proof ou prova.} \textcolor{purple}{Qual prova?} 

\newpage

Queremos provar que $f(\xi) = 0 \iff \varphi(\xi) = \xi$. Para isso, usaremos a forma geral da função de iteração $\varphi(x) = x + A(x)f(x)$.

%Provando primeiramente que a raiz da função é ponto fixo da função de iteração, $f(\xi) = 0 \Rightarrow \varphi(\xi) = \xi$. Começa-se calculando a função de iteração para a raiz $\xi$, isto é, $\varphi(\xi)$, obtendo então, pela forma geral, a raiz somada ao produto da função A pela função original $ \xi + A(\xi)f(\xi)$. Dado que $\xi$ é raiz de f(x), o valor da função calculado em $\xi$ é 0, portanto se tem $\xi + A(\xi) \cdot 0$; o que anula o produto com a função A, $\xi + 0$. Com isso, concluí-se que a função de iteração calculada na raíz é de fato igual a $\xi$.

Provando primeiramente que a raiz da função é ponto fixo da função de iteração $f(\xi) = 0 \Rightarrow \varphi(\xi) = \xi$
\begin{align*}
        \varphi(\xi) &= \xi + A(\xi)f(\xi), \ f(\xi) = 0 \\
        \varphi(\xi) &= \xi + A(\xi) \cdot 0 \\
        \varphi(\xi) &= \xi + 0\\
        \varphi(\xi) &= \xi
\end{align*}
Com isso, concluí-se que a função de iteração calculada na raíz da função f é de fato igual a $\xi$. \\

Interessa agora provar que se a função de iteração for calculada no ponto fixo dela, obtemos a raíz da função, $\varphi(\xi) = \xi \Rightarrow f(\xi) = 0$. 
\begin{align*}
        \varphi(\xi) &= \xi + A(\xi)f(\xi), \ f(\xi) = 0 \\
        \xi &= \xi + A(\xi) f(\xi) \\
        0 &= A(\xi) f(\xi), \ A(\xi) \neq 0 \\
        f(\xi) &= 0
\end{align*}
ou seja, o ponto fixo $\xi$ da função de iteração $\varphi$ é raíz da função original f, q.e.d. (\textit{quod erat demonstrandum}, qual estava-se a demonstrar).
%Partindo novamente da forma geral temos que o valor da $\varphi$ calculada em seu ponto fixo é o próprio $\xi$ somado ao produto das funções A calculada também em $\xi$ e a f avaliada nesse ponto, $\varphi(\xi) = \xi + A(\xi)f(\xi)$.  Dado que uma função avaliada em seu ponto fixo retorna o valor do próprio ponto fixo, pela igualdade temos que esse valor é equivalente ao ponto fixo mais o produto já descrito, $\xi = \xi + A(\xi)f(\xi)$.  Subtraindo $\xi$ dos dois lados da igualdade obtemos que o produto das funções A e original avaliadas no ponto fixo resulta em 0, $A(\xi)f(\xi) = 0$;  dado que é uma condição da forma que a função A avaliada no ponto fixo não pode ser nula, concluí-se que a função que transforma o produto em 0 é a f;  portanto, a função original avaliada no ponto fixo da função de iteração $\varphi$ é 0, 

\newpage

\section{Prova da convergência}

Dada uma função $\varphi(x)$ ela é chamada de função de iteração, pois $\varphi(x_k) = x_{k+1}$ e através dessas iterações podemos buscar o ponto fixo dela na qual $x_k = \xi$. Provaremos a seguir que dada uma raíz $\xi$ de f(x) isolada num intervalo I centrado na mesma, a sequência recursiva $\varphi(x_k) = x_{k+1}$ converge para $\xi$ tendo: a função contínua e derivável, e sua derivada também contínua, no intervalo (a,b); o módulo (M) de sua derivada limitado e inferior a unidade no intervalo ($|\varphi'(x)| \leq M < 1$ $\forall x \in (a, b)$); e o primeiro valor da iteração no intervalo ($x_0 \in I$).
\begin{align*}
    x_{k+1} &= \varphi(x_k) \\
    x_{k+1} - \xi &= \varphi(x_k) - \xi, \ \text{como $\xi$ é ponto fixo de $\varphi$} \qquad (1)\\
    x_{k+1} - \xi &= \varphi(x_k) - \varphi(\xi)
\end{align*}
pelo Teorema do Valor Médio (TVM) temos que
\begin{equation}
    \frac{\varphi(x_k) - \varphi(\xi)}{x_k - \xi} = \varphi'(c_k) \label{tvm}
\end{equation}
então
\begin{align*}
    x_{k+1} - \xi &= (x_k - \xi) \ \varphi'(c_k) \\
    |x_{k+1} - \xi| &= |(x_k - \xi) \ \varphi'(c_k)| \\
    |x_{k+1} - \xi| &= |x_k - \xi| \ |\varphi'(c_k)|, \ \text{e como $|\varphi'(x)| < 1$} \\
    |x_k - \xi| \ |\varphi'(c_k)| &< |x_k - \xi| \\
    |x_{k+1} - \xi| &< |x_k - \xi| 
\end{align*}

No limite, $x_k = \xi$, como pode-se verificar pela seguinte sequência de cálculos. Começando por retomar (1) obter-se-á que $|x_1 - \xi| \leq M \ |x_0 - \xi|$
\begin{align*}
    |x_1 - \xi| &= |\varphi(x_0) - \xi| \\
    \text{Pelo TVM,} \ \varphi(x_0) - \xi &= (x_0 - \xi) \ \varphi'(c_0) \text{, com $c_0 \in (x_0, \xi)$} \\
    |x_1 - \xi| &= |x_0 - \xi| \ |\varphi'(c_0)| \\
    \text{Como $|\varphi'(x)| \leq M$,} \ |\varphi'(c_0)| \ |x_0 - \xi| &\leq M \ |x_0 - \xi| \\
    |x_1 - \xi| &\leq M \ |x_0 - \xi|
\end{align*}
O próximo passo é obter o semelhante com $x_2$ e $x_1$, então generalizar isso para $x_k$, concluindo que $|x_k - \xi| \leq M^k \ |x_0 - \xi|$
\begin{align*}
    |x_2 - \xi| &= |\varphi(x_1) - \xi| \\
    |\varphi(x_1) - \xi| &= |x_1 - \xi| \ |\varphi'(c_1)| \text{, com $c_1 \in (x_1, \xi)$} \\
    |x_2 - \xi| &= |x_1 - \xi| \ |\varphi'(c_1)| \\
    \text{$|\varphi'(x)| \leq M$, então} \ |\varphi'(c_1)| \ |x_1 - \xi| &\leq M \ |x_1 - \xi| \\
    |x_2 - \xi| &\leq M \ |x_1 - \xi| \\
    \text{$|x_1 - \xi| \leq M \ |x_0 - \xi|$,} \ |x_2 - \xi| &\leq M^2 \ |x_0 - \xi| \\ % <= → |x2-csi| <= M | |x1-csi| <= M |x0 - csi|
    \vdots \\
    |x_k - \xi| &= |\varphi(x_{k-1}) - \xi| \\
    |\varphi(x_{k-1}) - \xi| &= |x_{k-1} - \xi| \ |\varphi'(c_{k-1})| \text{, com $c_{k-1} \in (x_{k-1}, \xi)$} \\
    |x_k - \xi| &= |x_{k-1} - \xi| \ |\varphi'(c_{k-1})| \\
    \text{$|\varphi'(x)| \leq M$, então} \ |\varphi'(c_{k-1})| \ |x_{k-1} - \xi| &\leq M \ |x_{k-1} - \xi|\\
    |x_k - \xi| &\leq M \ |x_{k-1} - \xi| \\
    \text{$|x_{k-1} - \xi| \leq M \ |x_{k-2} - \xi|$,} \ |x_k - \xi| &\leq M^2 \ |x_{k-2} - \xi| \\ 
    \text{$|x_k - \xi| \leq \ ... \ \leq M^k \ |x_0 - \xi|$,} \ |x_k - \xi| &\leq M^k \ |x_0 - \xi|
\end{align*}

\newpage

Enfim aplica-se o limite para obter que $x_k$ converge para a raiz $\xi$ de f, (tendo $0 < M < 1$)
\[ \lim_{k \to \infty} |x_k - \xi| \leq M^k \ |x_0 - \xi|\]
\[ \lim_{k \to \infty} |x_k - \xi| \leq 0 \ |x_0 - \xi|\]
\[ \lim_{k \to \infty} |x_k - \xi| = 0 \] logo,
\begin{equation}
    \lim_{k \to \infty} x_k = \xi \label{conv.mpf}
\end{equation}

    
%--------------------------------------------------------------------
\section{Método de Newton-Raphson}
