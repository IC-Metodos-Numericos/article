\documentclass[a0,portrait]{a0poster}

\usepackage{palatino}
\usepackage{epsfig}
\usepackage[T1]{fontenc}
\usepackage[utf8]{inputenc}
\usepackage[brazil]{babel}
\usepackage[usenames,dvipsnames]{xcolor}
\usepackage{amsmath,amsthm,amsfonts}
\usepackage{graphicx}
\usepackage{eso-pic}


% ----------------------------------------------------
% Paleta em azul p\'etroleo escuro
% ----------------------------------------------------
\definecolor{PetroleoEscuro}{RGB}{1,52,82}
\definecolor{Laranja}{RGB}{200, 95, 30}
% ----------------------------------------------------
% Macro para t\'itulos de se\c c\~ao com espa\c camento "aberto" (op\c c\~ao C)
% ----------------------------------------------------
\newcommand{\SectionHeader}[1]{%
  \begin{center}
    \parbox{15cm}{%
      \begin{center}
        {\color{Laranja}\rule[0.3cm]{15cm}{0.5cm}}\\[0.4cm]
        {\bfseries\LARGE #1}\\[0.3cm]
        {\color{Laranja}\rule[0.0cm]{15cm}{0.5cm}}
      \end{center}
    }
  \end{center}
}

\newcommand{\FaixaPatrocinadoresFigura}{%
  \AddToShipoutPictureBG*{%
    \AtPageLowerLeft{%
      \raisebox{1.5cm}{% distancia do rodapé
        \hspace*{2cm}% margem esquerda
        \includegraphics[width=80cm,height=6cm]{rodape_rede.png}%
      }%
    }%
  }%
}

\begin{document}
\FaixaPatrocinadoresFigura
\small

% ----------------------------------------------------
% Faixa superior / logo
% ----------------------------------------------------
\begin{center}
  \hspace{-2cm}
  \includegraphics[width=80cm,height=12cm]{faixa.png}
\end{center}

% ----------------------------------------------------
% T\'itulo do evento
% ----------------------------------------------------
\vspace{1cm}
\begin{center}
  \parbox{70cm}{
    \textcolor{Gray}{\rule[0cm]{70cm}{0.6cm}}\\[0.2cm]
    {\color{PetroleoEscuro}\Huge \bfseries \centering
      Explorando alguns efeitos dos erros de Ponto Flutuante\\[0.3cm]
    }
    \textcolor{Gray}{\rule[0cm]{70cm}{0.6cm}}
  }
\end{center}

% ----------------------------------------------------
% Autores
% ----------------------------------------------------
\vspace{1cm}
\begin{center}
{\LARGE \bf
  Ribas, E. R. L. D.$^{1,*}$, Pazini, D. S.$^2$, D`Afonseca, L. A.$^3$, Rocha, L. M.$^4$\\[0.5cm]
  Departamento de Matemática, Centro Federal de Educação Tecnológica de Minas Gerais (CEFET-MG)\\[0.5cm]
  $^{*}$Contato: enzorochaleitedinizribas@gmail.com
}
\end{center}

\vspace{1.5cm}

% ====================================================
%                      3 COLUNAS
% ====================================================

\noindent
% -------------------- COLUNA 1 ----------------------
\begin{minipage}[t]{0.31\textwidth}
  \raggedright\large

  % INTRODUCAO
  \SectionHeader{INTRODU\c C\~AO}

  \parbox{24cm}{
    {\LARGE\textsf{
      A \textit{aritmética de ponto flutuante} é o sistema adotado por computadores para que lidem com números reais utilizando uma notação compacta e eficaz. Essa técnica é utilizada para representar e manipular números reais de forma prática e eficiente. Ela permite representar números de grandezas diversas, que não podem ser armazenados com precisão, utilizando apenas números inteiros.
      A aritmética de ponto flutuante é amplamente utilizada em diversas áreas, como computação científica, gráficos de computador, simulações numéricas e processamento de sinais. No entanto, é importante compreender suas limitações e os possíveis erros que podem ocorrer durante as operações aritméticas, a fim de garantir resultados precisos e confiáveis em cálculos numéricos.
    }}
  }
   % ARITMETICA DE PF
   % TODO - otimizar o espaço, talvez transformando a lista em texto.
  \SectionHeader{Sistema de Ponto Flutuante}
  \parbox{24cm}{
    {\LARGE\textsf{
      Um sistema de ponto flutuante $F$ pode ser definido como
      \[
      F(\beta, t, L, U)\]
      cuja representação normalizada de um número real N nesse sistema é dada por
      \begin{equation}
      N = \pm (d_{1}.d_{2} . . . d_{t})_\beta \times \beta^e 
      \end{equation}
      em que
      \begin{itemize}
        \item \( N \) é o número real;
        \item \(\beta\) é a base que a máquina opera;
        \item \( t \) é o número de dígitos na mantissa, tal que \( 0 \leq d_{j} \leq \beta-1 \), j = 1, ...,t, \(d_{1} \neq 0\);
        \item \( L \) é o menor expoente inteiro;
        \item \( U \) é o maior expoente inteiro;
        \item \( e \) é o expoente inteiro no intervalo [\( L \),\( U \)].
      \end{itemize}
    }}
  }

  \vspace{1.5cm}

  % OBJETIVOS
  \SectionHeader{OBJETIVOS}

  \parbox{24cm}{
    {\LARGE\textsf{
      Lorem ipsum dolor sit amet, consectetur adipiscing elit.
      Proin sodales urna a vehicula dapibus. Cras tristique fermentum dolor a congue.
    }}
  }

  \vspace{1cm}

  {\LARGE
    \begin{eqnarray}
      X(\xi_N) &=& \sum_{i=1}^N \sum_{j=1}^N (x_i - x_j)^2 +
      \int_{\theta} f(\theta)\,d\theta.
      \nonumber
    \end{eqnarray}
  }

\end{minipage}
\hfill
% -------------------- COLUNA 2 ----------------------
\begin{minipage}[t]{0.31\textwidth}
  \raggedright\large

  % METODOLOGIA
  \SectionHeader{METODOLOGIA}

  \parbox{24cm}{
    {\LARGE\textsf{
      Lorem ipsum dolor sit amet, consectetur adipiscing elit.
      Nam laoreet feugiat purus, nec tempor elit blandit in.
      Integer volutpat sem sit amet laoreet elementum.\\[0.8cm]
      Mauris interdum congue lacus, sed auctor lectus ultrices et.
      Donec quis blandit ligula. Sed tempus justo eget tortor semper
      tristique. Maecenas faucibus nisi at sem suscipit, vel porta
      turpis feugiat.
    }}
  }

  \vspace{1.5cm}

  % RESULTADOS
  \SectionHeader{RESULTADOS}

  \vspace{1cm}

  % figura 1
  \begin{center}
    \includegraphics[height=8cm]{images/NumError1_mplt}
  \end{center}

  \parbox{24cm}{
    {\Large\textsf{
      \textbf{Figura 1:} Exemplo de legenda fict\'icia usando texto lorem
      ipsum apenas para compor o painel.
    }}
  }

  \vspace{1.8cm}

  % figura 2
  \begin{center}
    \includegraphics[height=8cm]{images/NumError2_mplt}
  \end{center}

  \parbox{24cm}{
    {\Large\textsf{
      \textbf{Figura 2:} Outra legenda ilustrativa descrevendo uma varia\c c\~ao
      em duas condi\c c\~oes gen\'ericas.
    }}
  }

  % figura 3
  \begin{center}
    \includegraphics[height=8cm]{images/NumError3_doubleplt_mplt.png}
  \end{center}

  \parbox{24cm}{
    {\Large\textsf{
      \textbf{Figura 3:} Outra legenda ilustrativa descrevendo uma varia\c c\~ao
      em duas condi\c c\~oes gen\'ericas.
    }}
  }

  % figura 4
  \begin{center}
    \includegraphics[height=8cm]{images/NumError4_mplt.png}
  \end{center}

  \parbox{24cm}{
    {\Large\textsf{
      \textbf{Figura 4:} Outra legenda ilustrativa descrevendo uma varia\c c\~ao
      em duas condi\c c\~oes gen\'ericas.
    }}
  }

\end{minipage}
\hfill
% -------------------- COLUNA 3 ----------------------
\begin{minipage}[t]{0.31\textwidth}
  \raggedright\large

  % TABELA
  \parbox{24cm}{
    {\Large\textsf{
      \textbf{Tabela 1:} Exemplo de tabela com estimativas de par\^ametros
      em um modelo gen\'erico.
    }}
  }

  \vspace{0.8cm}

  \begin{center}
    \begin{tabular}{ccc}
      \hline
      Par\^ametro & Estimativa & Valor p \\
      \hline
      $\beta_0$ & 1.23 & 0.12 \\
      $\beta_1$ & 0.98 & $< 0.001$ \\
      $\beta_2$ & 0.51 & 0.03 \\
      \hline
    \end{tabular}
  \end{center}

  \vspace{2cm}

  % CONCLUS\~AO
  \SectionHeader{CONCLUS\~AO}

  \parbox{24cm}{
    {\LARGE\textsf{
      Lorem ipsum dolor sit amet, consectetur adipiscing elit.
      Vestibulum aliquam augue et convallis accumsan. Sed vitae
      tortor nec sapien volutpat venenatis. Integer congue nisi
      sed magna aliquet, vel faucibus est malesuada.
    }}
  }

  \vspace{1.5cm}

  % REFER\^ENCIAS
  \SectionHeader{REFER\^ENCIAS}

  \parbox{24cm}{
    {\LARGE
      [1] Chapra, S. C., Canale, R. P. Numerical Methods for Engineers. McGraw-Hill International Editions, 1985. \\[0.5cm]
      [2] IEEE Standard for Floating-Point Arithmetic. IEEE Std 754-2008, 2008. \\[0.5cm]
      [3] IEE Standard for Floating-Point Arithmetic. IEEE Std 754-2019, 2019. \\[0.5cm]
    }
  }

  \vspace{1.5cm}

  % AGRADECIMENTOS
  \SectionHeader{AGRADECIMENTOS}

  \parbox{24cm}{
    {\LARGE\textsf{
      Agradecemos ao CEFET-MG pelo apoio institucional disponibilizada para a realiza\c c\~ao deste trabalho.
      Agradecemos aos orientadores e colegas do grupo de pesquisa
      em M\'etodos Num\'ericos por suas contribui\c c\~es e
      discuss\~oes enriquecedoras.
    }}
  }

\end{minipage}

\end{document}


