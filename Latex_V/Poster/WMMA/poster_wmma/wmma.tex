\documentclass[a0,portrait]{a0poster}
\usepackage{palatino}
\usepackage{epsfig}
\usepackage[T1]{fontenc}
\usepackage[utf8]{inputenc}
\usepackage[brazil]{babel}
\usepackage[usenames,dvipsnames]{xcolor}
\usepackage{amsmath,amsthm,amsfonts}
\usepackage{graphicx}
\usepackage{eso-pic}
% ----------------------------------------------------
% Paleta em azul p\'etroleo escuro
% ----------------------------------------------------
\definecolor{PetroleoEscuro}{RGB}{1,52,82}
\definecolor{Laranja}{RGB}{200, 95, 30}
% ----------------------------------------------------
% Macro para t\'itulos de se\c c\~ao com espa\c camento "aberto" (op\c c\~ao C)
% ----------------------------------------------------
\newcommand{\SectionHeader}[1]{%
  \begin{center}
    \parbox{15cm}{%
      \begin{center}
        {\color{Laranja}\rule[0.3cm]{15cm}{0.5cm}}\\[0.4cm]
        {\bfseries\LARGE #1}\\[0.3cm]
        {\color{Laranja}\rule[0.0cm]{15cm}{0.5cm}}
      \end{center}
    }
  \end{center}
}
\newcommand{\FaixaPatrocinadoresFigura}{%
  \AddToShipoutPictureBG*{%
    \AtPageLowerLeft{%
      \raisebox{1.5cm}{% distancia do rodapé
        \hspace*{2cm}% margem esquerda
        \includegraphics[width=80cm,height=6cm]{rodape_rede.png}%
      }%
    }%
  }%
}
\begin{document}
\FaixaPatrocinadoresFigura
\small
% ----------------------------------------------------
% Faixa superior / logo
% ----------------------------------------------------
\begin{center}
  \hspace{-2cm}
  \includegraphics[width=80cm,height=12cm]{faixa.png}
\end{center}
% ----------------------------------------------------
% T\'itulo do evento
% ----------------------------------------------------
\vspace{1cm}
\begin{center}
  \parbox{70cm}{
    \textcolor{Gray}{\rule[0cm]{70cm}{0.6cm}}\\[0.2cm]
    {\color{PetroleoEscuro}\Huge \bfseries \centering
      Explorando alguns efeitos dos erros de Ponto Flutuante\\[0.3cm]
    }
    \textcolor{Gray}{\rule[0cm]{70cm}{0.6cm}}
  }
\end{center}
% ----------------------------------------------------
% Autores
% ----------------------------------------------------
\vspace{-.2cm}
\begin{center}
{\LARGE \bf
  Ribas, E. R. L. D.$^{1,*}$, Pazini, D. S.$^2$, D`Afonseca, L. A.$^3$, Rocha, L. M.$^4$\\[0.5cm]
  Departamento de Matemática, Centro Federal de Educação Tecnológica de Minas Gerais (CEFET-MG)\\[0.5cm]
  $^{*}$Contato: enzorochaleitedinizribas@gmail.com
}
\end{center}

\vspace{1.5cm}

% ====================================================
%                      3 COLUNAS
% ====================================================

\noindent
% -------------------- COLUNA 1 ----------------------
\begin{minipage}[t]{0.31\textwidth}
  \raggedright\large
  % INTRODUCAO
  \SectionHeader{Introdução}
  \parbox{24cm}{
    {
      A \textit{aritmética de ponto flutuante} é o sistema adotado por computadores para que lidem com números reais utilizando uma notação compacta e eficaz. Essa técnica é utilizada para representar e manipular números reais de forma prática e eficiente. Ela permite representar números de grandezas diversas, que não podem ser armazenados com precisão, utilizando apenas números inteiros. 
      
      A aritmética de ponto flutuante é amplamente utilizada em diversas áreas, como computação científica, gráficos de computador, simulações numéricas e processamento de sinais. No entanto, é importante compreender suas limitações e os possíveis erros que podem ocorrer durante as operações aritméticas, a fim de garantir resultados precisos e confiáveis em cálculos numéricos. Os resultados a seguir são derivados de estudos realizados em um projeto de iniciação científica.
    }
  }
   % ARITMETICA DE PF
  \SectionHeader{Ponto Flutuante}
  \parbox{24cm}{
    {
      Um sistema de ponto flutuante $F$ pode ser definido como
      \[
      F(\beta, t, L, U)\]
      cuja representação normalizada de um número real N nesse sistema é dada por
      \begin{equation}
      N = \pm (d_{1}.d_{2} . . . d_{t})_\beta \times \beta^e 
      \end{equation}
      em que
      \begin{itemize}
        \item \( N \) é o número real;
        \item \(\beta\) é a base que a máquina opera;
        \item \( t \) é o número de dígitos na mantissa, tal que \( 0 \leq d_{j} \leq \beta-1 \), j = 1, ...,t, \(d_{1} \neq 0\);
        \item \( L \) é o menor expoente inteiro;
        \item \( U \) é o maior expoente inteiro;
        \item \( e \) é o expoente inteiro no intervalo [\( L \),\( U \)].
      \end{itemize}
    }
  }
  \vspace{1cm}
   % METODOLOGIA
  \SectionHeader{Metodologia}
  \parbox{24cm}{
      Para investigar os efeitos dos erros de ponto flutuante, foram realizados experimentos computacionais utilizando diferentes configurações de precisão numérica. Através da linguagem de programação Python junto a bibliotecas NumPy e Scipy para melhor confiabilidade das operações e outras bibliotecas gráficas como Matplotlib, seaborn, plotly, foram implementados algoritmos que reproduzem instabilidades em operações aritméticas.
  }
\end{minipage}
\hfill
% -------------------- COLUNA 2 ----------------------
\begin{minipage}[t]{0.31\textwidth}
  \raggedright\large
  % RESULTADOS
  \SectionHeader{Fenômenos Observados}
  % ERRO 1
  
  \parbox{24cm}{
  Um erro comum neste sistema é o da perda de significância, ou cancelamento catastrófico, que ocorre quando a subtração de dois números resulta em um valor com menos dígitos significativos do que os números originais. Para ilustrar isso consideramos duas situações.
  
  Na Figura 1, a função $f(x) = x^{10} + 1 - x^{10}$, embora para $x \in \mathbb{R}$ o resultado seja igual a 1, ao efetuarmos os cálculos usando ponto flutuante, observamos um comportamento inesperado em que o valor correto é exibido apenas até um certo valor de $x$. Após esse valor observamos um intervalo em que ocorre uma oscilação caótica no resultado da função e, em seguida, observamos que a função assume o valor zero.
  }
  % figura 1
  \begin{center}
    \includegraphics[height=9cm]{images/NumError1_mplt}
  \end{center}
  \parbox{24cm}{
    {
      \textbf{Figura 1:} Perda de significância em precisão de 32 bits.
    }
  }
  \vspace{1cm}

  % % ERRO 2
  \parbox{24cm}{
    Já na Figura 2, consideramos a função $f(x) = \frac{(1+x)-1}{x}$, que matematicamente é igual a 1 para todo \(x \neq 0\). Contudo, ao calcular essa função utilizando ponto flutuante de 64 bits, para valores de $x$ muito pequenos, também observamos um comportamento caótico.
  % ERRO 1
  }
  % figura 4
  \begin{center}
    \includegraphics[height=9cm]{images/NumError4_mplt.png}
  \end{center}

  \parbox{24cm}{
    {
      \textbf{Figura 2:} Erro de cancelamento catastrófico em 64 bits.
    }
  }

  \vspace{1cm}
   \parbox{24cm}
    {
      % ERRO 2
      Entretanto, o gráfico da expressão expandida $q$, calculado no mesmo sistema de ponto flutuante, produz resultados caóticos.
      Um outro erro é o não cancelamento adequado de termos em expressões matematicamente equivalentes. A Figura 3 apresenta a comparação entre os resultados obtidos ao calcular $p(x) = (x-1)^{6}$ e sua forma expandida $q(x) = x^6-6x^5+15x^4-20x^3+15x^2-6x+1$. Nesta figura, o gráfico de $p$ exibe os resultados obtidos pelo cálculo da expressão fatorada utilizando ponto flutuante obtendo a curva esperada. 
      Note que, apesar de possivelmente surpreendente, esse fenômeno não invalida a utilidade do ponto flutuante, pois o erro observado é da ordem de $10^{-14}$.
    }

\end{minipage}
\hfill
% -------------------- COLUNA 3 ----------------------
\begin{minipage}[t]{0.31\textwidth}
  \raggedright\large
  \vspace{1cm}
  % figura 2
  \begin{center}
    \includegraphics[height=9cm]{images/NumError2_mplt}
  \end{center}
  \parbox{24cm}{
    {
      \textbf{Figura 3:} Expressões equivalentes.
    }
  }
  \vspace{1cm}

  % ERRO 3 DERIVADA NUMÉRICA
  
  \parbox{24cm}{
  Também observamos erros em expressões racionais onde o numerador e o denominador ficam pequenos simultaneamente.
  Utilizando Diferenças finitas para aproximar a derivada de uma função, ao diminuir o valor de \(h\), espera-se que a aproximação melhore. Mas devido aos erros de arredondamento em ponto flutuante, para h muito pequeno, o erro na aproximação começa a se comportar de forma caótica. A Figura 4 ilustra esse fenômeno para a função $f(x) = \frac{x^{3}}{3} - 3x + 3$ cuja derivada é dada por $f'(x) = x^{2} - 3$. O valor aproximado da derivada é calculado por 
  {
  \[
    \textit{$D_f(x,h)$} = \frac{\textit{f}(x+h) - f(x-h)}{2h}
  \]
  }
  A derivada numérica via diferença central é então comparada com o valor exato da derivada em \(x = 1\), e o erro é definido como
  {
  \[
    \textit{Erro}(h) = |f'(1) - D_f(1,h)|
  \]
  } 
  }   
  % figura 4
  \begin{center}
    \includegraphics[height=9cm]{images/NumError3_doubleplt_mplt.png}
  \end{center}
  \parbox{24cm}{
    {
      \textbf{Figura 4:} Erro na Derivada Numérica calculada pelo método de Diferenças Finitas em 32 bits.
    }
  }
  
  \vspace{1cm}
  % REFER\^ENCIAS
  \SectionHeader{Referências}
  \parbox{24cm}{
    {
      [1] Chapra, S. C., Canale, R. P. Numerical Methods for Engineers. McGraw-Hill International Editions, 1985. \\[0.5cm]
      [2] IEE Standard for Floating-Point Arithmetic. IEEE Std 754-2019, 2019. \\[0.5cm]
      [3] Faires J. D., Burden R. L. Numerical Analysis. 7th Edition. Brooks/Cole, 2001.\\[0.5cm]
      [4] Ruggiero, M. A. G., Lopes, V. L. R. Cálculo Numérico: Aspectos Teóricos e Computacionais. 2th Edition. Pearson. 2010.
    }
  }
\end{minipage}

\end{document}


