%==============================================================================
%== template for LATEX poster =================================================
%==============================================================================
%
%--A0 beamer slide-------------------------------------------------------------
\documentclass[final]{beamer}
\usepackage[orientation=portrait,size=a0,
            scale=1.25         % font scale factor
           ]{beamerposter}
           
\geometry{
  hmargin=2.5cm, % little modification of margins
}

%
\usepackage[utf8]{inputenc}

\linespread{1.15}
%
%==The poster style============================================================
\usetheme{sharelatex}

%==Title, date and authors of the poster=======================================
\title
[I Workshop de Verão do Doutorado Multicêntrico em Matemática de Minas Gerais, 03 a 05 de Fevereiro de 2026, UFMG, Minas Gerais, Brasil] % Conference
{ % Poster title
Explorando alguns efeitos dos erros de Ponto
Flutuante
}

\author{ % Authors
Ribas, E. R. L. D., Pazini, D. S., D`Afonseca, L. A., Rocha, L. M.
}
\institute
{
  Centro Federal de Educação Tecnológica de Minas Gerais (CEFET-MG)
}
\date{\today}



\begin{document}
%==============================================================================
\begin{multicols}{2}
%==============================================================================
%==The poster content==========================================================
%==============================================================================

\section{Introdução}

Texto em linguagem natural... 

\structure{Texto em linguagem natural... }

Texto em linguagem natural... 

\begin{equation}
\varphi = f(x)
\end{equation}

Texto em linguagem natural... 

Em Ref.~\cite{ref1}...
Em Refs.~\cite{ref1,ref2}...

\section{Resultados e Discussão}

Texto em linguagem natural... 

\vskip1ex
\begin{table}
\centering
\caption{dados de erro?}
\begin{tabular}{ccccc}
\hline\hline
1 & 2 & 3 & 4 & 5\\
\hline
aaa & bbb & ccc & ddd & eee\\
aaaa & bbbb & cccc & dddd & eeee\\
aaaaa & bbbbb & ccccc & ddddd & eeeee\\
aaaaaa & bbbbbb & cccccc & dddddd & eeeeee\\
1.000 & 2.000 & 3.000 & 4.000 & 5.000\\
\hline\hline
\end{tabular}
\end{table}
\vskip2ex

Texto em linguagem natural... 


\subsection{Subsection}

Texto em linguagem natural... 


\vskip1ex
\begin{figure}
\centering
\includegraphics[width=0.99\columnwidth]{logo.png}
\caption{Essa imagem representa os resultados da pesquisa}
\end{figure}
\vskip2ex

Texto em linguagem natural... 

\subsection{Subsection com um título muito demasiadamente exageradamente longo mesmo}

Texto em linguagem natural... 

\section{Sumário e conclusões}

Texto em linguagem natural... 


%==============================================================================
%==End of content==============================================================
%==============================================================================

%--References------------------------------------------------------------------

\subsection{Referências}

\begin{thebibliography}{99}

\bibitem{ref1} J.~Doe, Article name, \textit{Phys. Rev. Lett.}

@book{chapra2011numerical,
  title={Numerical methods for engineers},
  author={Chapra, Steven C and Canale, Raymond P and others},
  volume={1221},
  year={2011},
  publisher={Mcgraw-hill New York}
}

\bibitem{chapra} Chapra, S. C.; Canale, R. P. \textit{Numerical Methods for Engineers}. 7th ed.; McGraw-Hill International Editions: New York, 2011.

\bibitem{ref2} J.~Doe, J. Smith, Other article name, \textit{Phys. Rev. Lett.}

\bibitem{web} \url{http://www.google.pl}

\end{thebibliography}
%--End of references-----------------------------------------------------------

\end{multicols}

%==============================================================================
\end{document}
