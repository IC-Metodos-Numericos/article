%------------------------------------------------------------------------------%
\documentclass{pssbmac}

%------------------------------------------------------------------------------%
\usepackage[brazil]{babel}
\usepackage[utf8]{inputenc}
%------------------------------------------------------------------------------%
\usepackage[T1]{fontenc}
\usepackage{graphicx}
\usepackage{indentfirst}
\usepackage{amsmath, amsfonts, amssymb, amsthm}
%\usepackage{url}
\usepackage{csquotes}
% ref bibliográficas
\usepackage[backend=biber, style=numeric-comp, maxnames=50]{biblatex}
\addbibresource{refs.bib}
\DeclareTextFontCommand{\emph}{\boldmath\bfseries}
\DefineBibliographyStrings{brazil}{phdthesis = {Tese de doutorado}}
\DefineBibliographyStrings{brazil}{mathesis = {Disserta\c{c}\~{a}o de mestrado}}
\DefineBibliographyStrings{english}{mathesis = {Master dissertation}}

%------------------------------------------------------------------------------%

\usepackage{xcolor}
\usepackage[hidelinks]{hyperref}

%\usepackage[table,fixpdftex,hyperref]{xcolor}
%
%\usepackage{hyperref}
%
%\urlstyle{sf}
%
%\hypersetup{%
%  colorlinks = false,
%  linkcolor  = black,
%  urlcolor   = black,
%  citecolor  = black
%}

\renewcommand\fbox{\fcolorbox{lightgray}{white}}
\setlength{\fboxsep}{0pt}

\usepackage{subcaption}
\usepackage[bottom]{footmisc}
\usepackage[justification=centerlast]{caption}

\setcounter{topnumber}   {4}
\setcounter{bottomnumber}{4}
\setcounter{totalnumber}{10}
\renewcommand{\textfraction}     {0.15}
\renewcommand{\topfraction}      {0.85}
\renewcommand{\bottomfraction}   {0.70}
\renewcommand{\floatpagefraction}{0.66}
\setcounter   {dbltopnumber}           {2}
\renewcommand*{\dbltopfraction}        {1}
\renewcommand*{\dblfloatpagefraction}{0.9}
\raggedbottom

%------------------------------------------------------------------------------%
\begin{document}

\title{Classe \LaTeX{} para Criação de e-Books para o PROFMAT}

\author{
    {\large Luis A. D'Afonseca}\thanks{luis.dafonseca@cefetmg.com}\\
    {\small CEFET-MG, Belo Horizonte, MG} \\
}
\criartitulo
%------------------------------------------------------------------------------%

%------------------------------------------------------------------------------%
\begin{figure}[b]
  \vspace{-0.5\baselineskip}
    \centering
    \begin{subfigure}[t]{0.31\textwidth}
        \centering
        \fbox{\includegraphics[width=0.9\textwidth]{NiceBooklet_Manual-capa}}
        \caption{{\small Capa do manual da classe. Fonte: o próprio autor~\cite{APOSTILA}.}}
        \label{fig:capa}
    \end{subfigure}\hfill%
    \begin{subfigure}[t]{0.31\textwidth}
        \centering
        \fbox{\includegraphics[width=0.9\textwidth]{Pagina_exemplo_1}}
        \caption{{\small Exemplo de um teorema e sua demonstração.
        Fonte: Vieira \cite{PIERRE_FERMAT_VIEIRA:CEFET:2023}.}}
        \label{fig:pagina_teorema}
    \end{subfigure}\hfill%
    \begin{subfigure}[t]{0.31\textwidth}
        \centering
        \fbox{\includegraphics[width=0.9\textwidth]{Pagina_exemplo_2}}
        \caption{{\small Exemplo de uma definição e um exemplo.
        Fonte: Reis \cite{TAMISSON_SANTOS_REIS:CEFET:2022}.}}
        \label{fig:pagina_exemplo}
    \end{subfigure}
    \vspace{-2mm}
    \caption{{\small Capa e páginas de textos produzidos na classe \textit{NiceBooklet}.}}%
    \vspace{-0.1\baselineskip}
\end{figure}
%------------------------------------------------------------------------------%

\renewcommand{\baselinestretch}{0.97}

\LaTeX{}~\cite{mittelbach2004latex} é a ferramenta mais utilizada para escrever
textos matemáticos, devido, entre outros fatores, à sua qualidade tipográfica e
separação entre conteúdo e formatação.
Porém, ela é bastante intimidadora para principiantes.
Provavelmente, pela necessidade de descrever o texto
utilizando comandos sem ver imediatamente o resultado.
Outra dificuldade é a enorme quantidade de comandos, pacotes e opções disponíveis,
especialmente para a formatação do texto.

Essa dificuldade se tornou especialmente evidente na orientação de
dissertações para o PROFMAT. Nesse programa, os mestrandos precisam atuar como
professores de matemática durante a realização do curso.
Além disso, a matriz curricular prevê disciplinas em todos os semestres e
também nas férias de verão.
Essas características limitam significativamente o tempo que o discente pode
dedicar para seu trabalho de conclusão de curso.

Outra característica do PROFMAT é que os alunos devem criar um produto
educacional. Não há restrição sobre o tipo de produto que pode ser
criado, porém, os mais naturais são apostilas e textos paradidáticos.
Como o público alvo desses produtos são alunos ou professores do Ensino
Básico, a formatação e estilo típicos de uma dissertação de mestrado
são pouco atraentes, podendo até mesmo ser intimidadores.
Desejando facilitar a criação de apostilas e que elas fossem
visualmente atraentes para os alunos do Ensino Básico,
desenvolvi a classe \LaTeX{} \textit{NiceBooklet}~\cite{APOSTILA}.
A Figura \ref{fig:capa} mostra a capa do manual de uso da classe,
que foi escrito utilizando-a.

As Figuras \ref{fig:pagina_teorema} e \ref{fig:pagina_exemplo}
apresentam exemplos de textos escritos com a classe,
onde podemos ver que a formatação destaca trechos importantes
utilizando cores atraentes.
A primeira imagem mostra um teorema (destacado em azul) e sua demonstração
(marcada por uma linha cinza a esquerda). A segunda imagem
mostra uma definição (também em azul) e um exemplo do seu uso em um exercício
(destacado em amarelo).
Outra característica gráfica é o uso de fontes grandes (14pt) para facilitar
a leitura em formato digital.
A classe inclui os pacotes necessários para escrever o texto em
português, para os principais comandos matemáticos e inclusão de figuras.
Escondendo assim essa complexidade do autor.
Diversas tarefas são automatizadas pela classe, por exemplo:
\begin{itemize}
    \setlength\itemsep{0em}
    \item a formatação da capa a partir do fornecimento de uma imagem;

    \item a criação de \textit{links} de navegação no canto inferior direito de cada página,
    que levam para o sumário e para os inícios do capítulo atual, anterior e posterior;

    \item o deslocamento das respostas dos exercícios para um capítulo próprio
    no final do texto, com \textit{links} levando para a resposta e retornando;

    \item a criação de \textit{links} com QR code, utilizados para fornecer material
    complementar \textit{online};

    \item a criação de um índice remissivo.
\end{itemize}

A seguir apresento uma lista de dissertações que criaram apostilas
utilizando a classe descrita neste trabalho:
%
A. L. A. Gomes criou um material abordando a Matemática dos
investimentos financeiros
\cite{ANDERSON_LUIS_DE_AVILA_GOMES:CEFET:2024};
%
P. S. F. Vieira escreveu uma apostila para apresentar o conceito de
infinito para alunos do Ensino Médio
\cite{PIERRE_FERMAT_VIEIRA:CEFET:2023};
%
V. P. de Jesus desenvolveu uma apostila sobre a aplicação da
Matemática no cálculo de materiais na Construção Civil
\cite{VILMAR_PEREIRA_DE_JESUS:CEFET:2022};
%
T. S. Reis criou um jogo de RPG (Role-Playing Game) para ser
usado como ferramenta para o ensino de probabilidades
\cite{TAMISSON_SANTOS_REIS:CEFET:2022};
%
E. B. L. Santos escreveu uma apostila para o ensino de noções de
programação e métodos numéricos para o Ensino Médio
\cite{Edson:CEFET:2024}.
%
Os e-Books desses trabalhos,
a classe com sua documentação
e arquivo de configuração para autocompletamento no TeXstudio
estão disponíveis para download na página do autor~\cite{PAGINA_LAD}.

%------------------------------------------------------------------------------%
\vspace{-0.5\baselineskip}
\section*{Agradecimentos}
\vspace{-0.3\baselineskip}

O autor agradece ao CEFET-MG e à FAPEMIG
pelo apoio financeiro pela Chamada FAPEMIG 16/2024, Processo PCE-00114-25.

%------------------------------------------------------------------------------%
\vspace{-0.5\baselineskip}
%\renewcommand{\baselinestretch}{0.9}
\printbibliography
\end{document}
%------------------------------------------------------------------------------%