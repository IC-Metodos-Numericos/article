% ======= Anotações e marcações ===========
\usepackage{xcolor}
\usepackage{todonotes} % para comentários visuais na margem

% Comandos personalizados para comentários
\newcommand{\TODO}[1]{\todo[inline,color=yellow!90!white]{TODO: #1}}          % Amarelo claro yellow!20!white
\newcommand{\FIXME}[1]{\todo[inline,color=orange!30!white]{FIXME: #1}}         % Laranja claro orange!30!white
\newcommand{\NOTE}[1]{\todo[inline,color=green!20!white]{NOTE: #1}}        % Verde claro green!20!white
\newcommand{\HACK}[1]{\todo[inline,color=purple!20!white]{HACK: #1}}        % Roxo claro purple!20!white
\newcommand{\BUG}[1]{\todo[inline,color=red!30!white]{BUG: #1}}          % Vermelho claro red!30!white

% Comentários por autor
\newcommand{\EnzoR}[1]{\todo[inline,color=violet!20!white]{@EnzoR: #1}}
\newcommand{\LucasM}[1]{\todo[inline,color=cyan!20!white]{@LucasM: #1}}
\newcommand{\DanielP}[1]{\todo[inline,color=green!20!white]{@DanielP: #1}}
\newcommand{\LuisD}[1]{\todo[inline,color=gold!30!white]{@LuisD: #1}}

% ======= Comandos personalizados ===========
\newcommand{\R}{\mathbb{R}}
\newcommand{\highlight}[1]{\hl{#1}}