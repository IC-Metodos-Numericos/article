% ======= Anotações e marcações ===========
\usepackage{xcolor}
\usepackage{todonotes} % para comentários visuais na margem

% Comandos personalizados para comentários
\newcommand{\TODO}[1]{\todo[color=blue!20!white]{TODO: #1}}          % Azul claro
\newcommand{\FIXME}[1]{\todo[color=red!30!white]{FIXME: #1}}         % Vermelho claro
\newcommand{\NOTE}[1]{\todo[color=orange!30!white]{NOTE: #1}}        % Laranja claro
\newcommand{\HACK}[1]{\todo[color=yellow!40!black]{HACK: #1}}        % Amarelo escuro
\newcommand{\BUG}[1]{\todo[color=purple!20!white]{BUG: #1}}          % Roxo claro

% Comentários por autor
\newcommand{\EnzoR}[1]{\todo[color=violet!20!white]{@EnzoR: #1}}
\newcommand{\LucasM}[1]{\todo[color=cyan!20!white]{@LucasM: #1}}
\newcommand{\Daniel}[1]{\todo[color=green!20!white]{@Daniel: #1}}
\newcommand{\LuisD}[1]{\todo[color=gold!30!white]{@LuisD: #1}}

% Uso inline (dentro de texto corrido)
\newcommand{\TODOinline}[1]{\todo[inline,color=blue!20!white]{TODO: #1}}
\newcommand{\FIXMEinline}[1]{\todo[inline,color=red!20!white]{FIXME: #1}}

% ======= Comandos personalizados ===========
\newcommand{\R}{\mathbb{R}}
\newcommand{\highlight}[1]{\hl{#1}}