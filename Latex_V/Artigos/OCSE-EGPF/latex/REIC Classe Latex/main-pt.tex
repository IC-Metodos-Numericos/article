\documentclass[portuguese]{sbc2025}%

%\usepackage{graphicx}% já está incluido na classe
%\usepackage[utf8]{inputenc} % obsoleto, desnecessário

\usepackage[misc,geometry]{ifsym} 

%%%%  %\usepackage{fontspec}

%% Os problemas com a classe eram originados por carregarem ambos os
%% pacotes fontenc e fontspec simultaneamente. Agora a classe detecta
%% qual engine está em uso e se for o pdflates então carrega o
%% fontenc. Se forem o xelatex ou lualatex então carrega o
%% fontspec. Eu particularemnte prefiro o lualatex ou o xetex


% \usepackage{fontawesome} %%% incluido na classe. Desnecessário
% chamá-lo aqui


%%%% \usepackage{academicons}
%%%% outro encrenqueiro que se tornou desnecessario. Deste typeface só
%%%% se usava o glifo do orcid e o codigo interno bombava.
%%%% Substitui pelo pacote orcidlink (carregado internamete na classe)
%%%% e consertei o codigo na classe.


%\usepackage{color} % carregada dentro da classe
%\usepackage{hyperref} % carregada dentro da classe

\usepackage{aas_macros}
\usepackage[bottom]{footmisc}

%\usepackage{supertabular}
%\usepackage{multicol}
%\usepackage{multirow}
%% O pacote tabularray é muito melhor para
% a criacao de tabela complexas e/ou loooongas; tudo de modo simples.
%% Torna o uso de multicol e multirow desnecessarios

\usepackage{tabularray}


\usepackage{afterpage}
\usepackage{url}
\usepackage{pifont}


\setcitestyle{square}


\definecolor{engtitle}{rgb}{0.5,0.5,0.5}
\definecolor{orcidlogo}{rgb}{0.37,0.48,0.13}
\definecolor{unilogo}{rgb}{0.16, 0.26, 0.58}
\definecolor{maillogo}{rgb}{0.58, 0.16, 0.26}
\definecolor{darkblue}{rgb}{0.0,0.0,0.0}
\hypersetup{colorlinks,breaklinks,
            linkcolor=darkblue,urlcolor=darkblue,
            anchorcolor=darkblue,citecolor=darkblue}
%\hypersetup{colorlinks,citecolor=blue,linkcolor=blue,urlcolor=blue}

%%%%%%% IMPORTANT: We disable hyperlinks by default with this line, to avoid the error "\pdfendlink ended up in different nesting level" while writing.
%\hypersetup{draft}



\begin{document}

\jid{REIC}
\jtitle{Revista Eletrônica de Iniciação Científica em Computação, 2025, XX:1}
\issn{1519-8219}
\doi{10.5753/reic.2025.XXXXXX}
\copyrightstatement{This work is licensed under a Creative Commons Attribution 4.0 International License}
\jyear{2025}

\category{Artigo de Pesquisa}
\title[O Computador Sempre Erra: Uma Exploração Gráfica dos Efeitos da Aritmética de Ponto Flutuante]{O Computador Sempre Erra: Uma Exploração Gráfica dos Efeitos da Aritmética de Ponto Flutuante}
\engtitle{\textcolor{engtitle}{The Computer Always Errs: A Graphical Exploration of Floating-Point Arithmetic Effects}}

%THE ORCID IS MANDATORY FOR EACH AUTHOR IN JBCS
\author[Ribas et al. 2025]{
\affil{\textbf{Enzo Rocha L. Diniz Ribas}~\orcidlink{0009-0002-4136-2182}~\textcolor{blue}{\faEnvelopeO}~~[{Centro Federal de Educação Tecnológica de Minas Gerais}~|\href{mailto:enzorochaleitedinizribas@gmail.com}{~{enzorochaleitedinizribas@gmail.com}}~]}

\affil{\textbf{Daniel Paz }~\orcidlink{0000-0000-0000-0000}~~[{Centro Federal de Educação Tecnológica de Minas Gerais}~|\href{mailto:danieldspazini3@gmail.com}{{\textit{danieldspazini3@gmail.com}}}~]}

% \affil{\textbf{Enzo Rocha L. Diniz Ribas}~\orcidlink{0009-0002-4136-2182}~\textcolor{blue}{\faEnvelopeO}~~[{Centro Universitário Dom Helder Câmara}~|\href{mailto:enzorochaleitedinizribas@gmail.com}{~{\textit{enzorochaleitedinizribas@gmail.com}}}~]}

% \affil{\textbf{Enzo Rocha L. Diniz Ribas}~\orcidlink{0009-0002-4136-2182}~\textcolor{blue}{\faEnvelopeO}~~[{Centro Universitário Dom Helder Câmara}~|\href{mailto:enzorochaleitedinizribas@gmail.com}{~{\textit{enzorochaleitedinizribas@gmail.com}}}~]}

}

%\begin{license}
%Published under the Creative Commons Attribution 4.0 International Public License (CC BY 4.0)
%\end{license}

\begin{frontmatter}

\maketitle

\begin{mail}
Centro Federal de Educação Tecnológica de Minas Gerais,  Av. Amazonas, 7675 - Nova Gameleira, Belo Horizonte - MG, 30510-000. 
\end{mail}

\begin{abstract-pt}
  PREVISAO: 500 a 750 palavras
  Este artigo tem como objetivo analisar quais são as principais vantagens e desvantagens de diversas bibliotecas e linguagens de programação utilizadas para visão computacional. A análise é feita com base em critérios como facilidade de uso, curva de aprendizado, comunidade de suporte, compatibilidade com diferentes plataformas, etc.. O estudo inclui uma revisão geral das principais bibliotecas e linguagens mais populares, como OpenCV, TensorFlow, YOLO, JavaCV, VisionWorks, BoofCV, OpenIMAJ, Halcon, SimpleCV entre outras, destacando suas características específicas e casos de uso ideais. Além disso, são apresentados exemplos práticos de aplicação dessas ferramentas em projetos reais, ilustrando como cada uma pode ser utilizada para resolver problemas comuns em visão computacional. A conclusão do artigo oferece recomendações para desenvolvedores e pesquisadores sobre a escolha da biblioteca ou linguagem mais adequada para suas necessidades específicas.
\end{abstract-pt}

\begin{abstract-en}
This text, formatted as a scientific article, aims to present the new SBC paper template, describing its main features and explaining how it should be used. This version, more specifically, should be used exclusively for articles written in Portuguese that will be published in any event proceeding series at SBC OpenLib. The abstract in Portuguese, as you can see in this example, must be before the abstract in English and must have between 500 and 750 words.
\end{abstract-en}

\begin{pchaves}
Mathematical Modeling, Scientific Computing, Numerical Analysis, Floating-Point Arithmetic, Last Place Units, Computational Math
\end{pchaves}

\begin{keywords}
Mathematical Modeling, Scientific Computing, Numerical Analysis, Floating-Point Arithmetic, Last Place Units, Computational Math
\end{keywords}

\begin{dates}
% This information will be provided by the editor before publishing the paper
\noindent{\sffamily\textbf{Recebido/Received:}} DD Month YYYY~~~$\bullet$~~~
{\sffamily\textbf{Aceito/Accepted:}} DD Month YYYY~~~$\bullet$~~~
{\sffamily\textbf{Publicado/Published:}} DD Month YYYY
\end{dates}


%\begin{license}
%Published under the Creative Commons Attribution 4.0 International Public License (CC BY 4.0)
%\end{license}

\end{frontmatter}
 

\section{Introdução}
\label{sec:intro}

Os computadores representam números reais usando a aritmética de ponto flutuante, que é uma aproximação finita e sistemática do conjunto continuo dos números reais (\cite{IEEE754_2019}). Embora essa representação seja eficiente e amplamente adotada, ela introduz erros numéricos inevitáveis devido à natureza discreta do armazenamento digital. Esses erros podem se manifestar de várias formas, como erros de arredondamento, cancelamento catastrófico, overflow e underflow, afetando a precisão e a confiabilidade dos cálculos computacionais.

Esse trabalho tem como objetivo fazer uma revisão da norma da IEEE (IEEE 754) sobre a aritmética de ponto flutuante e explorar graficamente os efeitos da aritmética de ponto flutuante em cálculos numéricos. Através de visualizações e experimentos controlados, buscamos ilustrar como esses erros se manifestam em diferentes situações.


Objetivos:
\begin{itemize}
    \item Revisar os conceitos fundamentais da aritmética de ponto flutuante conforme definido pela norma IEEE 754.
    \item Apresentar casos peculiares onde os erros de ponto flutuante se tornam evidentes.
    \item Explorar graficamente os efeitos dos erros numéricos introduzidos pela aritmética de ponto flutuante em cálculos comuns.
    \item Analisar como esses erros podem impactar a precisão e a confiabilidade dos resultados computacionais.
\end{itemize}

Contribuições principais:
\begin{itemize}
    \item Uma coleção de exemplos e gráficos que ilustram comportamentos típicos da aritmética de ponto flutuante.
    \item Análises interpretativas que relacionam os efeitos observados com os princípios teóricos da norma IEEE 754.
    \item Observações pedagógicas sobre como projetar experimentos numéricos robustos e interpretar resultados à luz dos erros de representação.
\end{itemize}

Organização do texto: na Seção~\ref{sec:fundamentos} revisamos conceitos fundamentais da representação em ponto flutuante; na Seção~\ref{sec:experimentos} apresentamos experimentos e visualizações; na Seção~\ref{sec:analise} discutimos implicações práticas e estratégias para mitigar erros; por fim, na Seção~\ref{sec:conclusao} sintetizamos os achados e sugerimos direções para trabalhos futuros.

A classe \textsl{sbc2025} é projetada para trabalhar com os \textit{engines} pdftex e luatex. Dessa forma, deve-se compilar o documento 


A classe \textsl{sbc2025} inclui internamente os seguintes pacotes:
\begin{itemize}
    \item xcolor
    \item graphicx
    \item amsmath amssymb
    \item hyperref
    \item babel
\end{itemize}
\noindent consequentemente, não há necessidade de incluí-los no preâmbulo.

{\bfseries Quem tentar compilar usando a opção \texttt{pdflatex} no Overleaf vai receber uma mensagem de erro solicitando o usuário a ajustar a opção de compilação.}

Para a pergunta \textit{Por que não funciona com o \texttt{pdflatex}}?  A resposta é: \textbf{fontes!} Pdflatex usa um esquema de codificação de fontes complexo. 
O fonte \texttt{academicons}\footnote{Do fonte em questão usa-se apenas o glifo associado ao Orchid.} não tem as definições necessárias para uso com o \texttt{pdflatex}. Enquanto isso não for realizado, \texttt{pdflatex} não pode ser usado. 




\section{Revisão Bibliográfica}



\subsection{Aritmética de Ponto Flutuante}
A norma IEEE 754 é um padrão amplamente adotado para a representação e manipulação de números em ponto flutuante em sistemas de hardware ou software. Ela define formatos de representação, regras de arredondamento, operações aritméticas e tratamento de exceções para garantir consistência e precisão nos cálculos numéricos. A norma especifica diferentes formatos de ponto flutuante, como simples precisão (32 bits) e dupla precisão (64 bits), cada um com sua própria estrutura de bits para representar o sinal, o expoente e a mantissa do número. Além disso, a norma estabelece regras para operações aritméticas, como adição, subtração, multiplicação e divisão, incluindo o tratamento de casos especiais, como overflow, underflow, NaN (Not a Number) e infinitos (\cite{IEEE754_2019}). A norma IEEE 754 é fundamental para garantir a interoperabilidade entre diferentes sistemas computacionais e linguagens de programação, permitindo que cálculos numéricos sejam realizados de forma consistente e previsível.

A representação de um numero real em aritmética de ponto flutuante conforme definido pela norma IEEE 754 consistem tem 3 diferentes formatos, no primero formato a representação é composta por três componentes principais: o sinal, o expoente e a mantissa (ou significando). O expoente da base utilizada. O bit de sinal indica se o número é positivo ou negativo. A mantissa representa a parte fracionária do número e é normalizada para garantir que o primeiro dígito seja sempre diferente de zero. A combinação desses três componentes permite representar uma ampla gama de números reais, incluindo números muito grandes e muito pequenos, com uma precisão limitada devido ao número finito de bits disponíveis para cada componente. O segundo formato é a represetação de infinitos, e por ultimo o terceiro formato é a representação de NaN's, que é usada para indicar resultados indefinidos ou inválidos em operações aritméticas.
\begin{itemize}
    \item Valor = $(-1)^{\text{sinal}} \times \text{mantissa} \times \text{base}^{\text{expoente}}$
    \item $\infty$ e $-\infty$
    \item qNaN (quiet NaN) e sNaN (signaling NaN)
\end{itemize}

qNaN e sNaN são dois tipos de Not-a-Number (NaN) definidos pela norma IEEE 754, usados para representar resultados indefinidos em aritmética de ponto flutuante.

Um conjunto de finitos pontos flutuantes representaveis a partir de um formato de ponto flutuante é denominado \textbf{conjunto de números em ponto flutuante}, e e definido pelos seguintes parametros:

\begin{itemize}
    \item $b$  = \textbf{Base}: A base do sistema numérico utilizado, 2 ou 10 (geralmente 2 em computadores).
    \item $p$ = \textbf{Precisão}: O número de dígitos na mantissa (ou significando) que determina a precisão dos números representados.
    \item $emax$ = \textbf{Expoente máximo}: O maior valor do expoente que pode ser representado.
    \item $emin$ = ($-emax + 1$) = \textbf{Expoente mínimo}: O menor valor do expoente que pode ser representado.
\end{itemize}

A norma IEEE 754 também define regras para arredondamento, que determinam como os números são aproximados quando não podem ser representados exatamente no formato de ponto flutuante. As principais estratégias de arredondamento sao:
\begin{itemize}
    \item roundToIntegralTiesToEven (Arredondar para o inteiro mais próximo, com empates arredondados para o número par mais próximo)
    \item roundToIntegralTorwardsZero (Arredondar para o inteiro mais próximo em direção a zero)
    \item roundToIntegralTowardPositive (Arredondar para o inteiro mais próximo em direção ao infinito positivo)
    \item roundToIntegralTowardNegative (Arredondar para o inteiro mais próximo em direção ao infinito negativo)
    \item roundToIntegralTiesToAway (Arredondar para o inteiro mais próximo, com empates arredondados para longe de zero)
    \item roundToIntegralExact (Arredondar exatamente)
\end{itemize}

\subsubsection{Units Of Last Place (ULP) e Quantum}
Unit Of Last Place (ULP) ou Unidade do Último Lugar é uma medida usada em aritmética de ponto flutuante para quantificar a precisão relativa de um número representado. A ULP representa a menor diferença entre dois números consecutivos que podem ser representados no formato de ponto flutuante. Em outras palavras, a ULP indica o "tamanho do passo" entre números representáveis próximos a um determinado valor como definido por John Harrison em \cite{HarrisonMachineCheckedFP}.

Na pratica podemos entender a ULP, segundo \cite{Muller2016} como: 
Se \(x\) é exatamente representável em um formato de ponto flutuante e não é uma potência inteira da base \(\beta\), o termo \(\mathrm{ulp}(x)\) (unidade do último lugar) denota a magnitude do último dígito do significando de \(x\). Ou seja, se
\[
x = \pm d_0 . d_1 d_2 \ldots d_{p-1} \times \beta^{e_x},
\]
então a unidade do último lugar de \(x\) é dada por
\[
\mathrm{ulp}(x) = \beta^{e_x - p + 1}.
\]
onde \(p\) é a precisão (número de dígitos no significando) e \(e_x\) é o expoente de \(x\).



A figura \ref{fig:float_density} ilustra a densidade dos números representáveis em ponto flutuante ao longo da reta numérica. Note que a densidade é maior próximo de zero e diminui à medida que nos afastamos de zero, refletindo a natureza relativa da precisão em aritmética de ponto flutuante.
\begin{figure}[htbp]
\centering
\includegraphics[width=\columnwidth]{anexos/a_cap2/floatingPointSystem.png}
\caption{Densidade dos números em ponto flutuante ao longo da reta numérica.}
\label{fig:float_density}
\end{figure}

Quando um número real \(x\) não tem representação exata em um formato de ponto flutuante, a ULP pode ser usada para medir o erro de arredondamento ao aproximar \(x\) pelo número representável mais próximo. O erro de arredondamento é frequentemente expresso em termos de ULPs, indicando quantas unidades do último lugar o valor aproximado difere do valor real.
\begin{figure}[htbp]
\centering
\includegraphics[width=\columnwidth]{anexos/a_cap2/rounding.png}
\caption{Erro de arredondamento de Pontos Flutuantes.}
\label{fig:float_rounding}
\end{figure}

O conceito de Quantum em aritmética de ponto flutuante é semelhante ao de ULP, mas com uma definição ligeiramente diferente. O Quantum é definido pela (\cite{IEEE754_2008}) como: O Quantum de uma representação finita em ponto flutuante é o valor de uma unidade na última posição do seu significando. Isso é igual à base (radix) elevada ao expoente \(q\), que é usado quando o significando é considerado como um inteiro.

Tanto o \emph{Quantum} quanto o \emph{ULP} medem a diferença entre dois números consecutivos de ponto flutuante, mas usam critérios ligeiramente diferentes nas fronteiras dos blocos de expoente.

Seja a aritmética binária (base $b=2$) com precisão $p$ (número de bits do significando, incluindo o bit implícito). Para números com expoente $e$ (isto é, $x\in[2^e,2^{e+1})$) o Quantum é constante e dado por
\[
\text{Quantum}(e)=2^{\,e-p+1},
\]
ou seja, o espaçamento entre representáveis dentro desse mesmo intervalo de expoente.

A definição de ULP usada por John Harrison (e por várias definições formais de precisão) é a distância entre os dois representáveis adjacentes que envolvem um valor real $x$. Em particular:
\begin{itemize}
    \item se $x$ não é exatamente uma potência de dois, então a ULP de $x$ coincide com o Quantum do seu intervalo de expoente;
    \item se $x=2^e$ (fronteira entre blocos de expoente), o Quantum observa o bloco acima ($[2^e,2^{e+1})$) enquanto a ULP de Harrison usa o espaçamento imediatamente abaixo (do bloco $[2^{e-1},2^e)$). Assim
\[
\text{ULP}_{\text{Harrison}}(2^e)=2^{\,e-p}= \frac{1}{2}\,\text{Quantum}(e).
\]
\end{itemize}
Para análise de erro de arredondamento, é habitual usar ULP: o erro de arredondamento em uma operação corretamente arredondada é no máximo $0.5$ ULP.

\subsubsection{Exemplo numérico}
Para \(e=2\), \(p=4\) e \(b=2\) temos:
\[
\text{Quantum}(e)=2^{\,e-p+1}=2^{2-4+1}=2^{-1}=\tfrac{1}{2}=0.5,
\]
\[
\text{ULP}_{\text{Harrison}}(2^e)=2^{\,e-p}=2^{2-4}=2^{-2}=\tfrac{1}{4}=0.25
= \tfrac{1}{2}\,\text{Quantum}(e).
\]
Para \(x_{\text{real}} = 4.3\) e \(x_{\text{float}} = 4.5\), temos:
\begin{align*}
\text{erro absoluto} &= \lvert 4.3 - 4.5 \rvert = 0.2,\\[4pt]
\frac{\text{erro}}{\text{Quantum}} &= \frac{0.2}{0.5} = 0.4\ \text{Quantum},\\[4pt]
\frac{\text{erro}}{\mathrm{ULP}_{\!H}} &= \frac{0.2}{0.25} = 0.8\ \mathrm{ULP}_{\!H}.
\end{align*}

\begin{figure}[htbp]
\centering
\includegraphics[width=\columnwidth]{anexos/a_cap2/quantum_vs_ulp.png}
\caption{Comparação entre Quantum e ULP na representação de números em ponto flutuante.}
\label{fig:quantum_vs_ulp}
\end{figure}




\section{Exemplo de Título Nível 1 (Seção)}
Lorem ipsum dolor sit amet, consectetur adipiscing elit, sed do eiusmod tempor incididunt ut labore et dolore magna aliqua. Ut enim ad minim veniam, quis nostrud exercitation ullamco laboris nisi ut aliquip ex ea commodo consequat. Duis aute irure dolor in reprehenderit in voluptate velit esse cillum dolore eu fugiat nulla pariatur. Excepteur sint occaecat cupidatat non proident, sunt in culpa qui officia deserunt mollit anim id est laborum. Lorem ipsum dolor sit amet, consectetur adipiscing elit, sed do eiusmod tempor incididunt ut labore et dolore magna aliqua. Ut enim ad minim veniam, quis nostrud exercitation ullamco laboris nisi ut aliquip ex ea commodo consequat. Duis aute irure dolor in reprehenderit in voluptate velit esse cillum dolore eu fugiat nulla pariatur. Excepteur sint occaecat cupidatat non proident, sunt in culpa qui officia deserunt mollit anim id est laborum.

\subsection{Exemplo de Título Nível 2 (Subseção)}

Lorem ipsum dolor sit amet, consectetur adipiscing elit, sed do eiusmod tempor incididunt ut labore et dolore magna aliqua. Ut enim ad minim veniam, quis nostrud exercitation ullamco laboris nisi ut aliquip ex ea commodo consequat. Duis aute irure dolor in reprehenderit in voluptate velit esse cillum dolore eu fugiat nulla pariatur. Excepteur sint occaecat cupidatat non proident, sunt in culpa qui officia deserunt mollit anim id est laborum.

\begin{quotation}
\textit{This is a longer quotation. Lorem ipsum dolor sit amet, consectetur adipiscing elit, sed do eiusmod tempor incididunt ut labore et dolore magna aliqua. Lorem ipsum dolor sit amet, consectetur adipiscing elit, sed do eiusmod tempor incididunt ut labore et magna aliqua. 
}\end{quotation} 

\subsubsection{Exemplo de Título Nível 3 (Subsubseção)}

Lorem ipsum dolor sit amet, consectetur adipiscing elit, sed do eiusmod tempor incididunt ut labore et dolore magna aliqua. Ut enim ad minim veniam, quis nostrud exercitation ullamco laboris nisi ut aliquip ex ea commodo consequat. Duis aute irure dolor in reprehenderit in voluptate velit esse cillum dolore eu fugiat nulla pariatur. Excepteur sint occaecat cupidatat non proident, sunt in culpa qui officia deserunt mollit anim id est laborum. Lorem ipsum dolor sit amet, consectetur adipiscing elit, sed do eiusmod tempor incididunt ut labore et dolore magna aliqua. Ut enim ad minim veniam, quis nostrud exercitation ullamco laboris nisi ut aliquip ex ea commodo consequat. Duis aute irure dolor in reprehenderit in voluptate velit esse cillum dolore eu fugiat nulla pariatur. Excepteur sint occaecat cupidatat non proident, sunt in culpa qui officia deserunt mollit anim id est laborum.

\paragraph{Exemplo de Título Nível 4 (Parágrafo).}

Lorem ipsum dolor sit amet, consectetur adipiscing elit, sed do eiusmod tempor incididunt ut labore et dolore magna aliqua. Ut enim ad minim veniam, quis nostrud exercitation ullamco laboris nisi ut aliquip ex ea commodo consequat. Duis aute irure dolor in reprehenderit in voluptate velit esse cillum dolore eu fugiat nulla pariatur. Excepteur sint occaecat cupidatat non proident, sunt in culpa qui officia deserunt mollit anim id est laborum. 
Let
\[
x=(x_1,\dots,x_n)\in R^n
\]be an \(n\)-dimensional vector. The sparse PCA problem can be written as
\begin{equation}\label{eq1}
\max\limits_{x}\{x^TAx-\rho\|x\|_0:x^Tx=1\},
\end{equation}
Lorem ipsum dolor sit amet, consectetur adipiscing elit, sed do eiusmod tempor incididunt ut labore et dolore magna aliqua. Ut enim ad minim veniam, quis nostrud exercitation ullamco laboris nisi ut aliquip ex ea commodo consequat. Duis aute irure dolor in reprehenderit in voluptate velit esse cillum dolore eu fugiat nulla pariatur\break equation (\ref{eq1}). 
\[\max\limits_{x}\{x^TAx :x^Tx=1\}.\]
If $A$ Lorem ipsum dolor sit amet, consectetur adipiscing elit, sed do eiusmod tempor incididunt ut labore et dolore magna aliqua. Ut enim ad minim veniam, quis nostrud exercitation ullamco laboris nisi ut aliquip ex ea commodo consequat. 


Equation (\ref{eq1}) is a special case of the following sparse generalized eigenvector problem (GEV):
\begin{equation}\label{GEV}
\max\limits_{x}\{x^TAx-\rho \|x\|_0: x^TBx\leq 1\},
\end{equation}

Lorem ipsum dolor sit amet, consectetur adipiscing elit, sed do eiusmod tempor incididunt ut labore et dolore magna aliqua. Ut enim ad minim veniam, quis nostrud exercitation ullamco laboris nisi ut aliquip ex ea commodo consequat. Duis aute irure dolor in reprehenderit in voluptate velit esse cillum dolore eu fugiat nulla pariatur. Excepteur sint occaecat cupidatat non proident, sunt in culpa qui officia deserunt mollit anim id est laborum. 

Lorem ipsum dolor sit amet, consectetur adipiscing elit, sed do eiusmod tempor incididunt ut labore et dolore magna aliqua. Ut enim ad minim veniam, quis nostrud exercitation ullamco laboris nisi ut aliquip ex ea commodo consequat. Duis aute irure dolor in reprehenderit in voluptate velit esse cillum dolore eu fugiat nulla pariatur. Excepteur sint occaecat cupidatat non proident, sunt in culpa qui officia deserunt mollit anim id est laborum \textbf{Table~\ref{tab2}}.



\section{Exemplo de Título Nível 1}

Lorem ipsum dolor sit amet, consectetur adipiscing elit, sed do eiusmod tempor incididunt ut labore et dolore magna aliqua. Ut enim ad minim veniam, quis nostrud exercitation ullamco laboris nisi ut aliquip ex ea commodo consequat. Duis aute irure dolor in reprehenderit in voluptate velit esse cillum dolore eu fugiat nulla pariatur. Excepteur sint occaecat cupidatat non proident, sunt in culpa qui officia deserunt mollit anim id est laborum. Lorem ipsum dolor sit amet, consectetur adipiscing elit, sed do eiusmod tempor incididunt ut labore et dolore magna aliqua. Ut enim ad minim veniam, quis nostrud exercitation ullamco laboris nisi ut aliquip ex ea commodo consequat. Duis aute irure dolor in reprehenderit in voluptate velit esse cillum dolore eu fugiat nulla pariatur. Excepteur sint occaecat cupidatat non proident, sunt in culpa qui officia deserunt mollit anim id est laborum.


\begin{table*}
\caption{Exemplo de legenda de tabela.  Duis aute irure dolor in reprehenderit in voluptate velit esse cillum dolore eu fugiat nulla pariatur. Excepteur sint occaecat cupidatat non proident, sunt in culpa qui officia deserunt mollit.} 
\centering
\begin{tabular*}{\textwidth}{@{}c\x c\x c\x c\x c\x c\x c\x c\x c\x c\x c\x c@{}}
\hline \hline
 Número   &  Vel (km/h)   & $\alpha$ (m/s$^2$)    &  $\epsilon^{(1)}$  &  $\epsilon^{(2)}$ 
         & $\delta^{(1)}$ & $\delta^{(2)}$  &  $\delta^{(3)}$    & $\gamma^{(1)}$ 
         & $\gamma^{(2)}$ & $\alpha^{(1)}$  & $\alpha^{(2)}$ \\
%
\hline
 1 & 3.5 & 2.0 & 0.20 & -0.05 & 0.00 & -0.20 & -0.05 & 0.20 & ~0.05 & 20 & 10 \\ 
 2 & 2.5 & 1.5 & 0.10 & -0.15 & 0.05 & -0.15 & ~0.00 & 0.10 & -0.05 & 20 & 40 \\ 
 3 & 3.0 & 1.8 & 0.20 & -0.05 & 0.25 & ~0.05 & ~0.20 & 0.15 & ~0.00 & 20 & 70$^a$ \\
\hline \hline
\end{tabular*}\label{tab2}
\end{table*}

Lorem ipsum dolor sit amet, consectetur adipiscing elit, sed do eiusmod tempor incididunt ut labore et dolore magna aliqua. Ut enim ad minim veniam, quis nostrud exercitation ullamco laboris nisi ut aliquip ex ea commodo consequat. Duis aute irure dolor in reprehenderit in voluptate velit esse cillum dolore eu fugiat nulla pariatur. Excepteur sint occaecat cupidatat non proident, sunt in culpa qui officia deserunt mollit anim id est laborum. Lorem ipsum dolor sit amet, consectetur adipiscing elit, sed do eiusmod tempor incididunt ut labore et dolore magna aliqua. Ut enim ad minim veniam, quis nostrud exercitation ullamco laboris nisi ut aliquip ex ea commodo consequat. Duis aute irure dolor in reprehenderit in voluptate velit esse cillum dolore eu fugiat nulla pariatur. Excepteur sint occaecat cupidatat non proident, sunt in culpa qui officia deserunt mollit anim id est laborum.

\begin{itemize}
\item Esse é um exemplo de lista de tópicos. Lorem ipsum dolor sit amet, consectetur adipiscing elit, sed do eiusmod tempor incididunt.
\item Lorem ipsum dolor sit amet, consectetur adipiscing elit, sed do eiusmod tempor incididunt ut labore et dolore magna aliqua. Ut enim ad minim veniam.
\item Lorem ipsum dolor sit amet, consectetur adipiscing elit, sed do eiusmod tempor incididunt ut labore et dolore magna aliqua. Ut enim ad minim veniam.
\end{itemize}

Lorem ipsum dolor sit amet, consectetur adipiscing elit, sed do eiusmod tempor incididunt ut labore et dolore magna aliqua. Ut enim ad minim veniam, quis nostrud exercitation ullamco laboris nisi ut aliquip ex ea commodo consequat. Duis aute irure dolor in reprehenderit in voluptate velit esse cillum dolore eu fugiat nulla pariatur. Excepteur sint occaecat cupidatat non proident, sunt in culpa qui officia deserunt mollit anim id est laborum. Lorem ipsum dolor sit amet, consectetur adipiscing elit, sed do eiusmod tempor incididunt ut labore et dolore magna aliqua. Ut enim ad minim veniam, quis nostrud exercitation ullamco laboris nisi ut aliquip ex ea commodo consequat. Duis aute irure dolor in reprehenderit in voluptate velit esse cillum dolore eu fugiat nulla pariatur. Excepteur sint occaecat cupidatat non proident, sunt in culpa qui officia deserunt mollit anim id est laborum.

\begin{enumerate}%
\item Esse é um exemplo de lista numerada. Lorem ipsum dolor sit amet, consectetur adipiscing elit, sed do eiusmod tempor incididunt.
\item Lorem ipsum dolor sit amet, consectetur adipiscing elit, sed do eiusmod tempor incididunt ut labore et dolore magna aliqua. Ut enim ad minim veniam.
\item Lorem ipsum dolor sit amet, consectetur adipiscing elit, sed do eiusmod tempor incididunt ut labore et dolore magna aliqua. Ut enim ad minim veniam.
\end{enumerate}

Lorem ipsum dolor sit amet, consectetur adipiscing elit, sed do eiusmod tempor incididunt ut labore et dolore magna aliqua. Ut enim ad minim veniam, quis nostrud exercitation ullamco laboris nisi ut aliquip ex ea commodo consequat. Duis aute irure dolor in reprehenderit in voluptate velit esse cillum dolore eu fugiat nulla pariatur. Excepteur sint occaecat cupidatat non proident, sunt in culpa qui officia deserunt mollit anim id est laborum. Lorem ipsum dolor sit amet, consectetur adipiscing elit, sed do eiusmod tempor incididunt ut labore et dolore magna aliqua. Ut enim ad minim veniam, quis nostrud exercitation ullamco laboris nisi ut aliquip ex ea commodo consequat. Duis aute irure dolor in reprehenderit in voluptate velit esse cillum dolore eu fugiat nulla pariatur. Excepteur sint occaecat cupidatat non proident, sunt in culpa qui officia deserunt mollit anim id est laborum.

Lorem ipsum dolor sit amet, consectetur adipiscing elit, sed do eiusmod tempor incididunt ut labore et dolore magna aliqua. Ut enim ad minim veniam, quis nostrud exercitation ullamco laboris nisi ut aliquip ex ea commodo consequat. Duis aute irure dolor in reprehenderit in voluptate velit esse cillum dolore eu fugiat nulla pariatur. Excepteur sint occaecat cupidatat non proident, sunt in culpa qui officia deserunt mollit anim id est laborum. Lorem ipsum dolor sit amet, consectetur adipiscing elit, sed do eiusmod tempor incididunt ut labore et dolore magna aliqua. Ut enim ad minim veniam, quis nostrud exercitation ullamco laboris nisi ut aliquip ex ea commodo consequat. Duis aute irure dolor in reprehenderit in voluptate velit esse cillum dolore eu fugiat nulla pariatur. Excepteur sint occaecat cupidatat non proident, sunt in culpa qui officia deserunt mollit anim id est laborum.


\subsection{Exemplo de Título Nível 2}

Lorem ipsum dolor sit amet, consectetur adipiscing elit, sed do eiusmod tempor incididunt ut labore et dolore magna aliqua. Ut enim ad minim veniam, quis nostrud exercitation ullamco laboris nisi ut aliquip ex ea commodo consequat. Duis aute irure dolor in reprehenderit in voluptate velit esse cillum dolore eu fugiat nulla pariatur. Excepteur sint occaecat cupidatat non proident, sunt in culpa qui officia deserunt mollit anim id est laborum. Lorem ipsum dolor sit amet, consectetur adipiscing elit, sed do eiusmod tempor incididunt ut labore et dolore magna aliqua. Ut enim ad minim veniam, quis nostrud exercitation ullamco laboris nisi ut aliquip ex ea commodo consequat. Duis aute irure dolor in reprehenderit in voluptate velit esse cillum dolore eu fugiat nulla pariatur. Excepteur sint occaecat cupidatat non proident, sunt in culpa qui officia deserunt mollit anim id est laborum \textbf{Figure \ref{Fig1}}.

Lorem ipsum dolor sit amet, consectetur adipiscing elit, sed do eiusmod tempor incididunt ut labore et dolore magna aliqua. Ut enim ad minim veniam, quis nostrud exercitation ullamco laboris nisi ut aliquip ex ea commodo consequat. Duis aute irure dolor in reprehenderit in voluptate velit esse cillum dolore eu fugiat nulla pariatur. Excepteur sint occaecat cupidatat non proident, sunt in culpa qui officia deserunt mollit anim id est laborum. Lorem ipsum dolor sit amet, consectetur adipiscing elit, sed do eiusmod tempor incididunt ut labore et dolore magna aliqua. Ut enim ad minim veniam, quis nostrud exercitation ullamco laboris nisi ut aliquip ex ea commodo consequat. Duis aute irure dolor in reprehenderit in voluptate velit esse cillum dolore eu fugiat nulla pariatur. Excepteur sint occaecat cupidatat non proident, sunt in culpa qui officia deserunt mollit anim id est laborum.

Lorem ipsum dolor sit amet, consectetur adipiscing elit, sed do eiusmod tempor incididunt ut labore et dolore magna aliqua. Ut enim ad minim veniam, quis nostrud exercitation ullamco laboris nisi ut aliquip ex ea commodo consequat. Duis aute irure dolor in reprehenderit in voluptate velit esse cillum dolore eu fugiat nulla pariatur. Excepteur sint occaecat cupidatat non proident, sunt in culpa qui officia deserunt mollit anim id est laborum. Lorem ipsum dolor sit amet, consectetur adipiscing elit, sed do eiusmod tempor incididunt ut labore et dolore magna aliqua. Ut enim ad minim veniam, quis nostrud exercitation ullamco laboris nisi ut aliquip ex ea commodo consequat. Duis aute irure dolor in reprehenderit in voluptate velit esse cillum dolore eu fugiat nulla pariatur. Excepteur sint occaecat cupidatat non proident, sunt in culpa qui officia deserunt mollit anim id est laborum.

\begin{figure}
\begin{center}
\includegraphics[width=\columnwidth]{sol.jpg}
\caption{Exemplo de legenda de figura.}\label{Fig1}
\end{center}
\end{figure}

Lorem ipsum dolor sit amet, consectetur adipiscing elit, sed do eiusmod tempor incididunt ut labore et dolore magna aliqua. Ut enim ad minim veniam, quis nostrud exercitation ullamco laboris nisi ut aliquip ex ea commodo consequat. Duis aute irure dolor in reprehenderit in voluptate velit esse cillum dolore eu fugiat nulla pariatur. Excepteur sint occaecat cupidatat non proident, sunt in culpa qui officia deserunt mollit anim id est laborum. Lorem ipsum dolor sit amet, consectetur adipiscing elit, sed do eiusmod tempor incididunt ut labore et dolore magna aliqua. Ut enim ad minim veniam, quis nostrud exercitation ullamco laboris nisi ut aliquip ex ea commodo consequat. Duis aute irure dolor in reprehenderit in voluptate velit esse cillum dolore eu fugiat nulla pariatur. Excepteur sint occaecat cupidatat non proident, sunt in culpa qui officia deserunt mollit anim id est laborum.

Lorem ipsum dolor sit amet, consectetur adipiscing elit, sed do eiusmod tempor incididunt ut labore et dolore magna aliqua. Ut enim ad minim veniam, quis nostrud exercitation ullamco laboris nisi ut aliquip ex ea commodo consequat. Duis aute irure dolor in reprehenderit in voluptate velit esse cillum dolore eu fugiat nulla pariatur. Excepteur sint occaecat cupidatat non proident, sunt in culpa qui officia deserunt mollit anim id est laborum. Lorem ipsum dolor sit amet, consectetur adipiscing elit, sed do eiusmod tempor incididunt ut labore et dolore magna aliqua. Ut enim ad minim veniam, quis nostrud exercitation ullamco laboris nisi ut aliquip ex ea commodo consequat. Duis aute irure dolor in reprehenderit in voluptate velit esse cillum dolore eu fugiat nulla pariatur. Excepteur sint occaecat cupidatat non proident, sunt in culpa qui officia deserunt mollit anim id est laborum \textbf{Figure \ref{Fig2}}.

\begin{figure*}
\begin{center}
\includegraphics[width=30pc]{sol.jpg}
\caption{{Exemplo de legenda de figura.}}
 \label{Fig2}
\end{center}
\end{figure*}

Lorem ipsum dolor sit amet, consectetur adipiscing elit, sed do eiusmod tempor incididunt ut labore et dolore magna aliqua. Ut enim ad minim veniam, quis nostrud exercitation ullamco laboris nisi ut aliquip ex ea commodo consequat. Duis aute irure dolor in reprehenderit in voluptate velit esse cillum dolore eu fugiat nulla pariatur. Excepteur sint occaecat cupidatat non proident, sunt in culpa qui officia deserunt mollit anim id est laborum. Lorem ipsum dolor sit amet, consectetur adipiscing elit, sed do eiusmod tempor incididunt ut labore et dolore magna aliqua. Ut enim ad minim veniam, quis nostrud exercitation ullamco laboris nisi ut aliquip ex ea commodo consequat. Duis aute irure dolor in reprehenderit in voluptate velit esse cillum dolore eu fugiat nulla pariatur. Excepteur sint occaecat cupidatat non proident, sunt in culpa qui officia deserunt mollit anim id est laborum.

Lorem ipsum dolor sit amet, consectetur adipiscing elit, sed do eiusmod tempor incididunt ut labore et dolore magna aliqua. Ut enim ad minim veniam, quis nostrud exercitation ullamco laboris nisi ut aliquip ex ea commodo consequat. Duis aute irure dolor in reprehenderit in voluptate velit esse cillum dolore eu fugiat nulla pariatur. Excepteur sint occaecat cupidatat non proident, sunt in culpa qui officia deserunt mollit anim id est laborum. Lorem ipsum dolor sit amet, consectetur adipiscing elit, sed do eiusmod tempor incididunt ut labore et dolore magna aliqua. Ut enim ad minim veniam, quis nostrud exercitation ullamco laboris nisi ut aliquip ex ea commodo consequat. Duis aute irure dolor in reprehenderit in voluptate velit esse cillum dolore eu fugiat nulla pariatur. Excepteur sint occaecat cupidatat non proident, sunt in culpa qui officia deserunt mollit anim id est laborum.

Lorem ipsum dolor sit amet, consectetur adipiscing elit, sed do eiusmod tempor incididunt ut labore et dolore magna aliqua. Ut enim ad minim veniam, quis nostrud exercitation ullamco laboris nisi ut aliquip ex ea commodo consequat. Duis aute irure dolor in reprehenderit in voluptate velit esse cillum dolore eu fugiat nulla pariatur. Excepteur sint occaecat cupidatat non proident, sunt in culpa qui officia deserunt mollit anim id est laborum. Lorem ipsum dolor sit amet, consectetur adipiscing elit, sed do eiusmod tempor incididunt ut labore et dolore magna aliqua. Ut enim ad minim veniam, quis nostrud exercitation ullamco laboris nisi ut aliquip ex ea commodo consequat. Duis aute irure dolor in reprehenderit in voluptate velit esse cillum dolore eu fugiat nulla pariatur. Excepteur sint occaecat cupidatat non proident, sunt in culpa qui officia deserunt mollit anim id est laborum.



\paragraph{Exemplo de Título Nível 4.}

Excepteur sint occaecat cupidatat non proident, sunt in culpa qui officia deserunt mollit anim id est laborum. Lorem ipsum dolor sit amet, consectetur adipiscing elit, sed do eiusmod tempor incididunt ut labore et dolore magna aliqua. Ut enim ad minim veniam, quis nostrud exercitation ullamco laboris nisi ut aliquip ex ea commodo consequat. Duis aute irure dolor in reprehenderit in voluptate velit esse cillum dolore eu fugiat nulla pariatur. 

Lorem ipsum dolor sit amet, consectetur adipiscing elit, sed do eiusmod tempor incididunt ut labore et dolore magna aliqua. Ut enim ad minim veniam, quis nostrud exercitation ullamco laboris nisi ut aliquip ex ea commodo consequat. Excepteur sint occaecat cupidatat non proident, sunt in culpa qui officia deserunt mollit anim id est laborum. Lorem ipsum dolor sit amet, consectetur adipiscing elit, sed do eiusmod tempor incididunt ut labore et dolore magna aliqua. Ut enim ad minim veniam, quis nostrud exercitation ullamco laboris nisi ut aliquip ex ea commodo consequat. Duis aute irure dolor in reprehenderit in voluptate velit esse cillum dolore eu fugiat nulla pariatur. Excepteur sint occaecat cupidatat non proident, sunt in culpa qui officia deserunt mollit anim id est laborum.


\input{capitulos/05-discussao.tex}

\input{capitulos/06-conclusao.tex}

\begin{declarations}

\begin{acknowledgements}
ESTA DECLARAÇÃO É OPCIONAL. Este é um texto de agradecimentos com várias linhas. Lorem ipsum dolor sit amet, consectetur adipiscing elit, sed do eiusmod tempor incididunt ut labore et dolore magna aliqua. Ut enim ad minim veniam, quis nostrud exercitation ullamco laboris nisi ut aliquip ex ea commodo consequat.
\end{acknowledgements}

\begin{funding}
ESTA DECLARAÇÃO É OPCIONAL. Esta pesquisa foi financiada por lorem ipsum dolor sit amet, consectetur adipiscing elit.
\end{funding}

\begin{contributions}
ESTA DECLARAÇÃO É OBRIGATÓRIA. Sugerimos que os autores descrevam sua contribuição usando a Taxonomia CRediT (\href{https://credit.niso.org/}{https://credit.niso.org/}) como neste exemplo: JV contribuiu para a concepção deste estudo. CB, RP e CM realizaram os experimentos. JV é o principal contribuidor e escritor deste manuscrito. Todos os autores leram e aprovaram o manuscrito final. 
\end{contributions}

\begin{interests}
ESTA DECLARAÇÃO É OBRIGATÓRIA. Se não houver conflitos de interesse, os autores devem declarar: ``Os autores declaram que não têm nenhum conflito de interesses''. Caso contrário, a declaração deve ser: ``Os autores declaram que têm os seguintes conflito de interesses: lorem ipsum dolor sit amet, consectetur adipiscing elit.''
\end{interests}

\begin{materials}
ESTA DECLARAÇÃO É OBRIGATÓRIA. 
  Se os autores estiverem disponibilizando seus dados e/ou códigos
  abertamente, a declaração deve ser: ``Os conjuntos de dados (e/ou
  softwares) gerados e/ou analisados durante o estudo atual estão
  disponíveis em \ldots''. Caso contrário, a declaração deve ser: ``Os
  conjuntos de dados (e/ou softwares) gerados e/ou analisados durante
  o estudo atual serão feitos mediante solicitação''.
\end{materials}

\begin{furtherinformation}
ESTA DECLARAÇÃO É DESEJÁVEL. Informações adicionais relevantes, como, por exemplo, a aprovação em comitê de ética ou o uso de ferramentas de IA generativa no desenvolvimento do artigo. Essa declaração é opcional, se não houver nada a ser acrescentado, pode ser deixada em branco
\end{furtherinformation}
\end{declarations}



\bibliographystyle{apalike-sol}
\bibliography{refs}

\end{document}
