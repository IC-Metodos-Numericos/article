
\section{Revisão Bibliográfica}

A revisão bibliográfica permitiu identificar as principais bibliotecas e linguagens de programação utilizadas em visão computacional, bem como suas características, vantagens e desvantagens. A seguir, são apresentadas as principais ferramentas analisadas:

\subsection{Visão computacional}
Visão computacional é o campo de estudo que busca permitir aos computadores descrever e interpretar o mundo físico. Ela envolve o desenvolvimento de algoritmos e técnicas de processamento, análise e extração de informações de imagens e vídeos. A visão computacional é amplamente utilizada em diversas aplicações, como reconhecimento facial (autenticação visual), detecção de objetos, análise de imagens médicas, modelagem 3D, entre outras \cite{Szeliski2022}.

\subsection{Linguagens de programação}
Diversas linguagens de programação são utilizadas para desenvolver aplicações de visão computacional. A seguir, são apresentadas algumas características das principais linguagens analisadas:
\begin{itemize}
    \item \textbf{Java}: Java é uma linguagem de programação que se destaca por ser multiplataforma, segura e orientada a objetos. \cite{Schildt2019}. Ela possui pequena gama de bibliotecas e frameworks para visão computacional, como JavaCV, BoofCV e OpenIMAJ, que permitem o desenvolvimento de aplicações nessa área.
    \item \textbf{Python}: Python é uma linguagem de programação gratuita amplamente utilizada em áreas da matemática, ciência de dados, inteligência artificial e visão computacional, devido à sua simplicidade, facilidade de aprendizado e utilização, portabilidade e vasta quantidade de bibliotecas. \cite{Lutz1999}. Ela possui um ecossistema robusto para processamento de imagens e visão computacional, com bibliotecas como OpenCV, TensorFlow e MediaPipe, que permitem o desenvolvimento de aplicações complexas de forma acessível e eficiente.
    \item \textbf{C}: 
    % C é uma linguagem de programação de baixo nível que oferece controle preciso sobre os recursos do sistema. Ela é amplamente utilizada em aplicações de visão computacional que exigem alto desempenho e eficiência, como processamento de imagens em tempo real e sistemas embarcados.
    \item \textbf{C++}: 
    % C++ é uma linguagem de programação de alto desempenho que é amplamente utilizada em aplicações de visão computacional que exigem processamento em tempo real. Ela oferece controle preciso sobre os recursos do sistema e é frequentemente utilizada em conjunto com bibliotecas como OpenCV para desenvolvimento de algoritmos de visão computacional.
\end{itemize}

\subsection{Bibliotecas de visão computacional}
Diversas bibliotecas e linguagens de programação são utilizadas para desenvolver aplicações de visão computacional. A seguir, são apresentadas algumas das principais ferramentas analisadas:
\begin{itemize}
    \item \textbf{OpenCV}:
    \item \textbf{TensorFlow}: 
    \item \textbf{YOLO (You Only Look Once)}: 
    \item \textbf{JavaCV}: 
    \item \textbf{VisionWorks}: 
    \item \textbf{BoofCV}: 
    \item \textbf{OpenIMAJ}: 
    \item \textbf{Halcon}: 
    \item \textbf{SimpleCV}: 
    
\end{itemize}

