
\section{Revisão Bibliográfica}



\subsection{Aritmética de Ponto Flutuante}
A norma IEEE 754 é um padrão amplamente adotado para a representação e manipulação de números em ponto flutuante em sistemas de hardware ou software. Ela define formatos de representação, regras de arredondamento, operações aritméticas e tratamento de exceções para garantir consistência e precisão nos cálculos numéricos. A norma especifica diferentes formatos de ponto flutuante, como simples precisão (32 bits) e dupla precisão (64 bits), cada um com sua própria estrutura de bits para representar o sinal, o expoente e a mantissa do número. Além disso, a norma estabelece regras para operações aritméticas, como adição, subtração, multiplicação e divisão, incluindo o tratamento de casos especiais, como overflow, underflow, NaN (Not a Number) e infinitos (\cite{IEEE754_2019}). A norma IEEE 754 é fundamental para garantir a interoperabilidade entre diferentes sistemas computacionais e linguagens de programação, permitindo que cálculos numéricos sejam realizados de forma consistente e previsível.

A representação de um numero real em aritmética de ponto flutuante conforme definido pela norma IEEE 754 consistem tem 3 diferentes formatos, no primero formato a representação é composta por três componentes principais: o sinal, o expoente e a mantissa (ou significando). O expoente da base utilizada. O bit de sinal indica se o número é positivo ou negativo. A mantissa representa a parte fracionária do número e é normalizada para garantir que o primeiro dígito seja sempre diferente de zero. A combinação desses três componentes permite representar uma ampla gama de números reais, incluindo números muito grandes e muito pequenos, com uma precisão limitada devido ao número finito de bits disponíveis para cada componente. O segundo formato é a represetação de infinitos, e por ultimo o terceiro formato é a representação de NaN's, que é usada para indicar resultados indefinidos ou inválidos em operações aritméticas.
\begin{itemize}
    \item Valor = $(-1)^{\text{sinal}} \times \text{mantissa} \times \text{base}^{\text{expoente}}$
    \item $\infty$ e $-\infty$
    \item qNaN (quiet NaN) e sNaN (signaling NaN)
\end{itemize}

qNaN e sNaN são dois tipos de Not-a-Number (NaN) definidos pela norma IEEE 754, usados para representar resultados indefinidos em aritmética de ponto flutuante.

Um conjunto de finitos pontos flutuantes representaveis a partir de um formato de ponto flutuante é denominado \textbf{conjunto de números em ponto flutuante}, e e definido pelos seguintes parametros:

\begin{itemize}
    \item $b$  = \textbf{Base}: A base do sistema numérico utilizado, 2 ou 10 (geralmente 2 em computadores).
    \item $p$ = \textbf{Precisão}: O número de dígitos na mantissa (ou significando) que determina a precisão dos números representados.
    \item $emax$ = \textbf{Expoente máximo}: O maior valor do expoente que pode ser representado.
    \item $emin$ = ($-emax + 1$) = \textbf{Expoente mínimo}: O menor valor do expoente que pode ser representado.
\end{itemize}

A norma IEEE 754 também define regras para arredondamento, que determinam como os números são aproximados quando não podem ser representados exatamente no formato de ponto flutuante. As principais estratégias de arredondamento sao:
\begin{itemize}
    \item roundToIntegralTiesToEven (Arredondar para o inteiro mais próximo, com empates arredondados para o número par mais próximo)
    \item roundToIntegralTorwardsZero (Arredondar para o inteiro mais próximo em direção a zero)
    \item roundToIntegralTowardPositive (Arredondar para o inteiro mais próximo em direção ao infinito positivo)
    \item roundToIntegralTowardNegative (Arredondar para o inteiro mais próximo em direção ao infinito negativo)
    \item roundToIntegralTiesToAway (Arredondar para o inteiro mais próximo, com empates arredondados para longe de zero)
    \item roundToIntegralExact (Arredondar exatamente)
\end{itemize}

\subsubsection{Units Of Last Place (ULP) e Quantum}
Unit Of Last Place (ULP) ou Unidade do Último Lugar é uma medida usada em aritmética de ponto flutuante para quantificar a precisão relativa de um número representado. A ULP representa a menor diferença entre dois números consecutivos que podem ser representados no formato de ponto flutuante. Em outras palavras, a ULP indica o "tamanho do passo" entre números representáveis próximos a um determinado valor como definido por John Harrison em \cite{HarrisonMachineCheckedFP}.

Na pratica podemos entender a ULP, segundo \cite{Muller2016} como: 
Se \(x\) é exatamente representável em um formato de ponto flutuante e não é uma potência inteira da base \(\beta\), o termo \(\mathrm{ulp}(x)\) (unidade do último lugar) denota a magnitude do último dígito do significando de \(x\). Ou seja, se
\[
x = \pm d_0 . d_1 d_2 \ldots d_{p-1} \times \beta^{e_x},
\]
então a unidade do último lugar de \(x\) é dada por
\[
\mathrm{ulp}(x) = \beta^{e_x - p + 1}.
\]
onde \(p\) é a precisão (número de dígitos no significando) e \(e_x\) é o expoente de \(x\).



A figura \ref{fig:float_density} ilustra a densidade dos números representáveis em ponto flutuante ao longo da reta numérica. Note que a densidade é maior próximo de zero e diminui à medida que nos afastamos de zero, refletindo a natureza relativa da precisão em aritmética de ponto flutuante.
\begin{figure}[htbp]
\centering
\includegraphics[width=\columnwidth]{anexos/a_cap2/floatingPointSystem.png}
\caption{Densidade dos números em ponto flutuante ao longo da reta numérica.}
\label{fig:float_density}
\end{figure}

Quando um número real \(x\) não tem representação exata em um formato de ponto flutuante, a ULP pode ser usada para medir o erro de arredondamento ao aproximar \(x\) pelo número representável mais próximo. O erro de arredondamento é frequentemente expresso em termos de ULPs, indicando quantas unidades do último lugar o valor aproximado difere do valor real.
\begin{figure}[htbp]
\centering
\includegraphics[width=\columnwidth]{anexos/a_cap2/rounding.png}
\caption{Erro de arredondamento de Pontos Flutuantes.}
\label{fig:float_rounding}
\end{figure}

O conceito de Quantum em aritmética de ponto flutuante é semelhante ao de ULP, mas com uma definição ligeiramente diferente. O Quantum é definido pela (\cite{IEEE754_2008}) como: O Quantum de uma representação finita em ponto flutuante é o valor de uma unidade na última posição do seu significando. Isso é igual à base (radix) elevada ao expoente \(q\), que é usado quando o significando é considerado como um inteiro.

Tanto o \emph{Quantum} quanto o \emph{ULP} medem a diferença entre dois números consecutivos de ponto flutuante, mas usam critérios ligeiramente diferentes nas fronteiras dos blocos de expoente.

Seja a aritmética binária (base $b=2$) com precisão $p$ (número de bits do significando, incluindo o bit implícito). Para números com expoente $e$ (isto é, $x\in[2^e,2^{e+1})$) o Quantum é constante e dado por
\[
\text{Quantum}(e)=2^{\,e-p+1},
\]
ou seja, o espaçamento entre representáveis dentro desse mesmo intervalo de expoente.

A definição de ULP usada por John Harrison (e por várias definições formais de precisão) é a distância entre os dois representáveis adjacentes que envolvem um valor real $x$. Em particular:
\begin{itemize}
    \item se $x$ não é exatamente uma potência de dois, então a ULP de $x$ coincide com o Quantum do seu intervalo de expoente;
    \item se $x=2^e$ (fronteira entre blocos de expoente), o Quantum observa o bloco acima ($[2^e,2^{e+1})$) enquanto a ULP de Harrison usa o espaçamento imediatamente abaixo (do bloco $[2^{e-1},2^e)$). Assim
\[
\text{ULP}_{\text{Harrison}}(2^e)=2^{\,e-p}= \frac{1}{2}\,\text{Quantum}(e).
\]
\end{itemize}
Para análise de erro de arredondamento, é habitual usar ULP: o erro de arredondamento em uma operação corretamente arredondada é no máximo $0.5$ ULP.

\subsubsection{Exemplo numérico}
Para \(e=2\), \(p=4\) e \(b=2\) temos:
\[
\text{Quantum}(e)=2^{\,e-p+1}=2^{2-4+1}=2^{-1}=\tfrac{1}{2}=0.5,
\]
\[
\text{ULP}_{\text{Harrison}}(2^e)=2^{\,e-p}=2^{2-4}=2^{-2}=\tfrac{1}{4}=0.25
= \tfrac{1}{2}\,\text{Quantum}(e).
\]
Para \(x_{\text{real}} = 4.3\) e \(x_{\text{float}} = 4.5\), temos:
\begin{align*}
\text{erro absoluto} &= \lvert 4.3 - 4.5 \rvert = 0.2,\\[4pt]
\frac{\text{erro}}{\text{Quantum}} &= \frac{0.2}{0.5} = 0.4\ \text{Quantum},\\[4pt]
\frac{\text{erro}}{\mathrm{ULP}_{\!H}} &= \frac{0.2}{0.25} = 0.8\ \mathrm{ULP}_{\!H}.
\end{align*}

\begin{figure}[htbp]
\centering
\includegraphics[width=\columnwidth]{anexos/a_cap2/quantum_vs_ulp.png}
\caption{Comparação entre Quantum e ULP na representação de números em ponto flutuante.}
\label{fig:quantum_vs_ulp}
\end{figure}

