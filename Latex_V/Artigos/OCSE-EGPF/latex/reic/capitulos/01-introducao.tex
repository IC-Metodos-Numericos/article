
\section{Introdução}
\label{sec:intro}

Observações iniciais do Prof. Ralha:
\begin{itemize}
    \item Este arquivo faz uso ``implícito"\ de texto em latim que é provido por um certo pacote---não vou citar o nome---que desenvolvedores de classes \LaTeX\ usam para facilitar testes. Vi que esse pacote não está incluido no preâmbulo deste documento. Ainda bem!
    \item por gentileza, entendam que publicar em jornais científicos está relacionado a tipografia profissional. Então deixem este aspecto para o \LaTeX. Seria aconselhável a leitura de textos em typografia para que entendam que ``editoração de textos'' feita por produtos comerciais tipo eh ... vocês sabem quais! não é nem de perto algo que se aproxime da tipografia!
    \item Por óbvio, deve-se sempre iniciar a criação de um arquivo \TeX\ pelo seu preâmbulo. Deve-se evitar \textit{tomar emprestado} templates de outras pessoas. Tenho observado que muitos ``usuários'' não observam este fato levando a casos em que o arquivo inclui o mesmo pacote duas, três e até quatro vezes.
\end{itemize}

A classe \textsl{sbc2025} é projetada para trabalhar com os \textit{engines} pdftex e luatex. Dessa forma, deve-se compilar o documento 
\begin{enumerate}
    \item no Overleaf, ajustando no menu opção de compilação para \textbf{xelatex} ou \textbf{lualatex}.\footnote{Os engines são denominados xetex e luatex. Já os comandos de compilação são \textsl{xelatex} e \textsl{lualatex}.}
    \item se compilando localmente, na sua IDE faça o ajuste do compilador. Se usuário raiz, no terminal de comando, digite \textsl{xelatex filename} ou \textsl{lualatex filename}. Não é necessário digitar a extensão \textsl{tex}.
\end{enumerate}

A classe \textsl{sbc2025} inclui internamente os seguintes pacotes:
\begin{itemize}
    \item xcolor
    \item graphicx
    \item amsmath amssymb
    \item hyperref
    \item babel
\end{itemize}
\noindent consequentemente, não há necessidade de incluí-los no preâmbulo.

{\bfseries Quem tentar compilar usando a opção \texttt{pdflatex} no Overleaf vai receber uma mensagem de erro solicitando o usuário a ajustar a opção de compilação.}

Para a pergunta \textit{Por que não funciona com o \texttt{pdflatex}}?  A resposta é: \textbf{fontes!} Pdflatex usa um esquema de codificação de fontes complexo. 
O fonte \texttt{academicons}\footnote{Do fonte em questão usa-se apenas o glifo associado ao Orchid.} não tem as definições necessárias para uso com o \texttt{pdflatex}. Enquanto isso não for realizado, \texttt{pdflatex} não pode ser usado. 
