
\section{Introdução}
\label{sec:intro}

Os computadores representam números reais usando a aritmética de ponto flutuante, que é uma aproximação finita e sistemática do conjunto continuo dos números reais (\cite{IEEE754_2019}). Embora essa representação seja eficiente e amplamente adotada, ela introduz erros numéricos inevitáveis devido à natureza discreta do armazenamento digital. Esses erros podem se manifestar de várias formas, como erros de arredondamento, cancelamento catastrófico, overflow e underflow, afetando a precisão e a confiabilidade dos cálculos computacionais.

Esse trabalho tem como objetivo fazer uma revisão da norma da IEEE (IEEE 754) sobre a aritmética de ponto flutuante e explorar graficamente os efeitos da aritmética de ponto flutuante em cálculos numéricos. Através de visualizações e experimentos controlados, buscamos ilustrar como esses erros se manifestam em diferentes situações.


Objetivos:
\begin{itemize}
    \item Revisar os conceitos fundamentais da aritmética de ponto flutuante conforme definido pela norma IEEE 754.
    \item Apresentar casos peculiares onde os erros de ponto flutuante se tornam evidentes.
    \item Explorar graficamente os efeitos dos erros numéricos introduzidos pela aritmética de ponto flutuante em cálculos comuns.
    \item Analisar como esses erros podem impactar a precisão e a confiabilidade dos resultados computacionais.
\end{itemize}

Contribuições principais:
\begin{itemize}
    \item Uma coleção de exemplos e gráficos que ilustram comportamentos típicos da aritmética de ponto flutuante.
    \item Análises interpretativas que relacionam os efeitos observados com os princípios teóricos da norma IEEE 754.
    \item Observações pedagógicas sobre como projetar experimentos numéricos robustos e interpretar resultados à luz dos erros de representação.
\end{itemize}

Organização do texto: na Seção~\ref{sec:fundamentos} revisamos conceitos fundamentais da representação em ponto flutuante; na Seção~\ref{sec:experimentos} apresentamos experimentos e visualizações; na Seção~\ref{sec:analise} discutimos implicações práticas e estratégias para mitigar erros; por fim, na Seção~\ref{sec:conclusao} sintetizamos os achados e sugerimos direções para trabalhos futuros.

A classe \textsl{sbc2025} é projetada para trabalhar com os \textit{engines} pdftex e luatex. Dessa forma, deve-se compilar o documento 


A classe \textsl{sbc2025} inclui internamente os seguintes pacotes:
\begin{itemize}
    \item xcolor
    \item graphicx
    \item amsmath amssymb
    \item hyperref
    \item babel
\end{itemize}
\noindent consequentemente, não há necessidade de incluí-los no preâmbulo.

{\bfseries Quem tentar compilar usando a opção \texttt{pdflatex} no Overleaf vai receber uma mensagem de erro solicitando o usuário a ajustar a opção de compilação.}

Para a pergunta \textit{Por que não funciona com o \texttt{pdflatex}}?  A resposta é: \textbf{fontes!} Pdflatex usa um esquema de codificação de fontes complexo. 
O fonte \texttt{academicons}\footnote{Do fonte em questão usa-se apenas o glifo associado ao Orchid.} não tem as definições necessárias para uso com o \texttt{pdflatex}. Enquanto isso não for realizado, \texttt{pdflatex} não pode ser usado. 

