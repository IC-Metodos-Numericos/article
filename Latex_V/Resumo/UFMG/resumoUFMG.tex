\documentclass[11pt,a4paper]{article}
\usepackage{times}
\usepackage{tgtermes}
\usepackage[T1]{fontenc}
\usepackage{amsmath,amssymb,latexsym, cite}
\usepackage{color,graphicx}
\usepackage[utf8]{inputenc}
\usepackage[brazilian]{babel}
\usepackage{fancyhdr}
\usepackage{setspace}
\usepackage{multicol}

\setstretch{1,15}


\newcommand{\R}{\mathbb {R}}
\newcommand{\N}{\mathbb {N}}
\newcommand{\F}{{\mathcal F}}
\newcommand{\W}{{\mathcal W}}
\newcommand{\T}{{\mathcal T}}
\renewcommand{\O}{{\mathcal O}}
\renewcommand{\S}{{\mathcal S}}
\renewcommand{\phi}{{\varphi}}
\newcommand{\dist}{{\rm dist}}
\newcommand{\Int}{\mathrm{Int}}
\newcommand{\Tr}{{\rm Tr}}
\renewcommand{\div}{{\rm div}}
\newcommand{\eps}{{\varepsilon}}
\renewcommand{\epsilon}{{\varepsilon}}
\renewcommand{\theta}{{\vartheta}}

\numberwithin{equation}{section}

\usepackage{titlesec}
\titleformat{\section}
  {\normalfont\fontsize{16}{20}\bfseries}{\thesection}{1em}{}
\titleformat{\subsection}
  {\normalfont\fontsize{16}{20}\bfseries}{\thesubsection}{1em}{}
\titlespacing*{\section}
{0pt}{2.ex plus 1ex minus .2ex}{.0ex}
\titlespacing*{\subsection}
{0pt}{0.ex}{.0ex}

\setlength{\parindent}{0pt}
\setlength{\parskip}{5pt plus 2pt minus 1 pt}
\topmargin  -12mm
\evensidemargin 5mm
\oddsidemargin  0mm
\textwidth  158mm
\textheight 245mm
\headheight 14pt
\headsep 1.2cm


\pagestyle{fancy}
\rfoot{}
\chead{}
\lfoot{
\textit{\hrule \hfill 
\\
I Workshop Mineiro de Matemática Aplicada}
}
\lhead{
 \textit{I Workshop de Verão do Doutorado Multicêntrico em Matemática de Minas Gerais}
 \rhead{
  \textit{03 a 05 de Fevereiro de 2026}}}
\begin{document}
%\pagestyle{empty}
\pagenumbering{gobble}
\begin{center}
\begin{spacing}{2.05}
{\fontsize{20}{20}
\bf
Explorando alguns efeitos dos erros de Ponto Flutuante
}
\end{spacing}
\end{center}
\vspace{-1.25cm}
\begin{center}
{\fontsize{14}{20}
\bf
E. R. L. D. Ribas\textsuperscript{\textdagger}, D. S. Pazini\textsuperscript{\textdaggerdbl}, L. A. D'Afonseca, L. M. Rocha }
\bigskip

{\fontsize{12}{20}
Centro Federal de Educação Tecnológica de Minas Gerais (CEFET-MG) \\
enzorochaleitedinizribas@gmail.com\textsuperscript{\textdagger}, danieldspazini3@gmail.com\textsuperscript{\textdaggerdbl}
}
\end{center}
\vspace{10pt}
\subsection*{Resumo}
Este trabalho apresenta resultados de estudos realizados em um projeto de iniciação científica cujo objetivo é explorar graficamente alguns efeitos do cálculos realizados com ponto flutuante.

Um sistema de ponto flutuante é a maneira utilizada pelos computadores para representar números reais através de uma notação compacta e eficaz, padronizado pelo IEEE. Ele permite escrever números de grandezas diversas utilizando apenas números inteiros. No entanto, essa representação fica limitada ao adequar o número ao sistema adotado, causando erros que podem se acumular em operações sucessivas, produzindo resultados imprecisos ou incorretos.

Um erro comum neste sistema é o da perda de significância, ou cancelamento catastrófico, que ocorre quando a subtração de dois números resulta em um valor com menos dígitos significativos do que os números originais.
Um exemplo desse comportamento é a função $f(x) = x^{10} + 1 - x^{10}$. Embora para $x \in \mathbb{R}$ o resultado é igual a 1, ao efetuarmos os cálculos usando ponto flutuante, observamos o comportamento ilustrado no gráfico da Figura 1, em que o valor correto é exibido apenas até um certo valor de $x$. Após esse valor observamos um intervalo em que ocorre uma oscilação caótica no resultado da função. Posteriormente, observamos que a função assume o valor zero.

\begin{multicols}{2}
\begin{center}
  \includegraphics[height=3cm,width=7.5cm]{imagens/erro1}
  Figura 1: Perda de significância.
\end{center}
\begin{center}
  \includegraphics[height=3cm,width=7.5cm]{imagens/error2zoomzoom}\\
  Figura 2: Expressões equivalentes.
\end{center}
\end{multicols}
\vspace{-0.5\baselineskip}

Um outro erro é o não cancelamento adequado de termos em expressões matematicamente equivalentes. A Figura 2 apresenta a comparação entre os resultados obtidos ao calcular $p(x) = (x-1)^{6}$ e sua forma expandida $q(x) = x^6-6x^5+15x^4-20x^3+15x^2-6x+1$. Nesta figura, o gráfico de $p$ exibe os resultados obtidos pelo cálculo da expressão fatorada utilizando ponto flutuante obtendo a curva esperada. Entretanto, o gráfico da expressão expandida $q$, calculado no mesmo sistema de ponto flutuante, produz resultados caóticos. Note que, apesar de possivelmente surpreendente, esse fenômeno não invalida a utilidade do ponto flutuante, pois o erro observado é da ordem de $10^{-14}$.

Os fenômenos observados nas figuras decorrem de erros de arredondamento inerentes ao sistema de ponto flutuante, destacando a importância da escolha das expressões matemáticas em implementações computacionais.

\end{document} 

