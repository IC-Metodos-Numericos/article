\documentclass[11pt,a4paper]{article}
\usepackage{times}
\usepackage{tgtermes}
\usepackage[T1]{fontenc}
\usepackage{amsmath,amssymb,latexsym, cite}
\usepackage{color,graphicx}
\usepackage[utf8]{inputenc}
\usepackage[brazilian]{babel}
\usepackage{fancyhdr}
\usepackage{setspace}
\usepackage{multicol}

\setstretch{1,15}


\newcommand{\R}{\mathbb {R}}
\newcommand{\N}{\mathbb {N}}
\newcommand{\F}{{\mathcal F}}
\newcommand{\W}{{\mathcal W}}
\newcommand{\T}{{\mathcal T}}
\renewcommand{\O}{{\mathcal O}}
\renewcommand{\S}{{\mathcal S}}
\renewcommand{\phi}{{\varphi}}
\newcommand{\dist}{{\rm dist}}
\newcommand{\Int}{\mathrm{Int}}
\newcommand{\Tr}{{\rm Tr}}
\renewcommand{\div}{{\rm div}}
\newcommand{\eps}{{\varepsilon}}
\renewcommand{\epsilon}{{\varepsilon}}
\renewcommand{\theta}{{\vartheta}}

\numberwithin{equation}{section}

\usepackage{titlesec}
\titleformat{\section}
  {\normalfont\fontsize{16}{20}\bfseries}{\thesection}{1em}{}
\titleformat{\subsection}
  {\normalfont\fontsize{16}{20}\bfseries}{\thesubsection}{1em}{}
\titlespacing*{\section}
{0pt}{2.ex plus 1ex minus .2ex}{.0ex}
\titlespacing*{\subsection}
{0pt}{0.ex}{.0ex}

\setlength{\parindent}{0pt}
\setlength{\parskip}{5pt plus 2pt minus 1 pt}
\topmargin  -12mm
\evensidemargin 5mm
\oddsidemargin  0mm
\textwidth  158mm
\textheight 245mm
\headheight 14pt
\headsep 1.2cm


\pagestyle{fancy}
\rfoot{}
\chead{}
\lfoot{
\textit{\hrule \hfill 
\\
I Workshop Mineiro de Matemática Aplicada}
}
\lhead{
 \textit{I Workshop de Verão do Doutorado Multicêntrico em Matemática de Minas Gerais}
 \rhead{
  \textit{03 a 05 de Fevereiro de 2026}}}


\begin{document}
%\pagestyle{empty}
\pagenumbering{gobble}

\begin{center}
\begin{spacing}{2.05}
{\fontsize{20}{20}
\bf
Explorando alguns efeitos dos erros de Ponto Flutuante
}
\end{spacing}
\end{center}
\vspace{-1.25cm}
\begin{center}
{\fontsize{14}{20}
\bf
E. R. L. D. Ribas\textsuperscript{\textdagger}, D. S. Pazini\textsuperscript{\textdaggerdbl}, L. A. D'Afonseca, L. M. Rocha }
\bigskip


{\fontsize{12}{20}
Centro Federal de Educação Tecnológica de Minas Gerais (CEFET-MG) \\
enzorochaleitedinizribas@gmail.com\textsuperscript{\textdagger}, danieldspazini3@gmail.com\textsuperscript{\textdaggerdbl}

}
\end{center}

\vspace{10pt}

Este trabalho apresenta estudos realizados em um projeto de iniciação científica do CEFET-MG.

\subsection*{Resumo}

Um sistema de ponto flutuante é o método utilziado por computadores para para representar números reais utilizando uma notação compacta e eficaz. Ele permite representar números de grandezas diversas, que não podem ser armazenados com precisão, utilizando apenas números inteiros.No entanto, essa representação possui limitações, o que pode levar a erros de arredondamento e perda de precisão em cálculos numéricos. Esses erros podem se acumular em operações sucessivas, resultando em resultados imprecisos ou incorretos.

A perda de significância (ou cancelamento catastrófico) ocorre de modo mais evidente quando há grande diferença de ordem de grandeza entre os números envolvidos na operação. Por exemplo, ao somar um número muito grande com um número muito pequeno, o número pequeno pode ser "esquecido" devido à falta de precisão na representação de ponto flutuante. Isso pode levar a resultados imprecisos ou incorretos. 

\begin{multicols}{2}
\begin{center}
  \includegraphics[height=3cm,width=8cm]{imagens/error2zoom} \\
  Figura 1: Exemplo de figura.
\end{center}
\begin{center}
  \includegraphics[height=3cm,width=8cm]{imagens/erro1}
\\
  Figura 2: Exemplo de figura.
\end{center}
\end{multicols}

Falar sobre os graficos / erros acima ( erro de perda de significancia e erro de arredondamento)

\end{document} 

